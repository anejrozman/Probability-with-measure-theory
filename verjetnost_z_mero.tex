\documentclass[a4paper,12pt]{article}
\usepackage[margin=3cm]{geometry}
\usepackage[slovene]{babel}
\usepackage[utf8]{inputenc}
\usepackage[T1]{fontenc}
\usepackage{lmodern}
\usepackage{url}
\usepackage{graphicx}
\usepackage{amsmath}
\usepackage{amssymb}
\usepackage{amsfonts}
\usepackage{hyperref}
\usepackage{amsthm}
\usepackage{pgfpages}
\usepackage{colortbl}
\usepackage{tikz}
\usepackage{array}
\usepackage{amsmath,amsthm, amsfonts,amssymb}
\usepackage{mathtools}
\usepackage{tikz}
\usepackage{dsfont}


% ukazi za matematicna okolja
\theoremstyle{definition} % tekst napisan pokoncno
\newtheorem{definicija}{Definicija}[section]
\newtheorem{primer}[definicija]{Primer}
\newtheorem{zgled}[definicija]{Zgled}
\newtheorem{opomba}[definicija]{Opomba}

\renewcommand\endprimer{\hfill$\diamondsuit$}


\theoremstyle{plain} % tekst napisan posevno
\newtheorem{lema}[definicija]{Lema}
\newtheorem{izrek}[definicija]{Izrek}
\newtheorem{trditev}[definicija]{Trditev}
\newtheorem{posledica}[definicija]{Posledica}

\newcommand{\R}{\mathbb{R}}
\newcommand{\N}{\mathbb{N}}
\newcommand{\E}{\mathbb{E}}
\newcommand{\F}{\mathcal{F}}
\newcommand{\A}{\mathcal{A}}
\newcommand{\sh}{š}
\newcommand{\ch}{č}
\newcommand{\zh}{ž}
\newcommand{\SH}{Š}
\newcommand{\CH}{Č}
\newcommand{\ZH}{Ž}




\begin{document}

\begin{titlepage}
    UNIVERZA V LJUBLJANI
  
    FAKULTETA ZA MATEMATIKO IN FIZIKO
  
    \vspace{0.5cm}
    Finančna matematika - 1. stopnja
  
    \begin{center}
        \vspace{7cm}
            Anej Rozman
  
        \vspace{0.4cm}
        \textbf{\Large{Verjetnost z mero}}
        \vspace{0.3cm}
  
        Zapiski po predavanjih doc.\@ dr.\@ Matije Vidmarja v študijskem letu 2023/2024.
    \end{center}
    \vfill
        Ljubljana, 2023     
    \thispagestyle{empty}
\end{titlepage}

\newpage
  
\tableofcontents
   
\newpage
    
\section{Mera}
    \subsection{Merljivost in mere}
        \subsubsection{Merjlive mnozice}
            \begin{definicija}
                Naj bo $\mathcal{A} \subset 2^\Omega$, torej $\mathcal{A} \in 2^{2^\Omega}$ Pravimo, da je $\mathcal{A}$ zaprta za:
                \begin{enumerate}
                    \item $c^\Omega$ (zaprta za komplemente v $\Omega$) $\iff \forall A \in \mathcal{A} \Rightarrow \Omega \backslash A \in \mathcal{A} $
                    \item $\cap $ (zaprta za preseke) $\iff A \cap A' \in \mathcal{A}$ kadarkoli $\{A, A'\} \subset \mathcal{A}$
                    \item $\cup$ (zaprta za unije) $\iff  A \cup A' \in \mathcal{A}$ kadarkoli $\{A, A'\} \subset \mathcal{A}$
                    \item $\backslash$ (zaprta za razlike) $\iff A \backslash A' \in \mathcal{A}$ kadarkoli $\{A, A'\} \subset \mathcal{A}$
                    \item $\sigma\cap$ (zaprta za stevne preseke) $\iff \cap_{n \in \N}A_n \in \mathcal{A}$ kadarkoli je $\left(A_n \right)_{n \in \N}$ zaporedje v $\mathcal{A}$
                    \item $\sigma\cup$ (zaprta za stevne unije) $\iff \cup_{n \in \N}A_n \in \mathcal{A} $ kadarkoli je $\left(A_n \right)_{n \in \N}$ zaporedje v $\mathcal{A}$
                
                \end{enumerate} 
            \end{definicija}
            
            \begin{definicija}($\sigma$-algebra, pod-$\sigma$-algebra in algebra)
                \begin{enumerate}
                    \item$\mathcal{A}$ je $\sigma$-algebra na $\Omega \iff (\Omega, \A)$ je merljiv prostor $\iff \emptyset \in \mathcal{A}$ in $\mathcal{A}$ je zaprta za $c^\Omega$ in $\sigma\cup$. Ce je $\mathcal{A} \ \sigma$-algebra na $\Omega$ potem: $A$ je $\mathcal{A}$-merljiva $\iff A \in \mathcal{A}.$
                    \item $\mathcal{B}$ je pod-$\sigma$-algebra $\mathcal{A} \iff \mathcal{B} \subset \mathcal{A}$ in $\mathcal{B}$ je $\sigma$-algebra na $\Omega$.
                    \item $\mathcal{A}$ je algebra na $\Omega \iff \emptyset \in \mathcal{A}$ in $\mathcal{A}$ je zaprta za $c^\Omega$ in $\cup$.
                \end{enumerate}
            \end{definicija}

        \begin{opomba}
            V primeru ko nimamo podane mnozice $\Omega$ lahko vzamemo $\Omega = \cup\A$ in velja $\A \subset 2^{\Omega}.$
        \end{opomba}

        \begin{zgled}
            $2^\Omega$ je $\sigma$-algebra na $\Omega$ in $\{\emptyset, \Omega\}$ je $\sigma$-algebra na $\Omega$. Klicemo ju diskretna in trivialna $\sigma$-algebra.
        \end{zgled}

        \begin{zgled}
        Naj bo $A \subset \Omega$. Potem je $ \sigma_\Omega A:=\{\emptyset, A, A^C, \Omega\}$ $\sigma$-algebra na $\Omega$.
        \end{zgled}

        \begin{zgled}
        $\sigma_\Omega^{ccc} := \{ A \in 2^\Omega: A \ \text{je stevna ali} \ \Omega \backslash A \ \text{je stevna}\}$ je $\sigma$-algebra na $\Omega$.
        To je oznaka za stevno kostevno $\sigma$-algebro na $\Omega$. Seveda je $\sigma_\Omega^{ccc}$ = $2^\Omega$ razen ce $\Omega$ ni stevna.
        \end{zgled}

        \begin{zgled}
        Naj bo $\mathcal{P}$ particija $\Omega$ (torej $\mathcal{P} \subset 2^\Omega$ in $\mathcal{P}$ je druzina paroma disjunktnih mnozic, ki pokrije $\Omega$).
        Potem je $\sigma \mathcal{P} := \{\cup R \mid R \subset P \ in \ (R \ ali \ P\backslash R \ je \ stevna)\}$ je sigma algebra na $\Omega$.
        \end{zgled}

        \begin{trditev}
            Naj bo $\A \subset 2^\Omega$ zaprta za $c^\Omega$ in naj bo $\emptyset \in \A$. Potem je $\A$ $\sigma$-algebra na $\Omega$ ce in samo ce je $\A$ zaprta za $\sigma\cup$, in v tem primeru je $\A$ zaprta za $\cap$ in $\cup$ in $\backslash$.
        \end{trditev}
        \begin{proof}
            Sledi iz de Morganovih zakonov:
            $$
            \cap_{n \in \N} A_n = \Omega \backslash \left(\cup_{n \in \N}(\Omega\backslash A_n) \right)
            $$
            $$
            \cup_{n \in \N} A_n = \Omega \backslash \left(\cap_{n \in \N}(\Omega\backslash A_n) \right)
            $$
            Zaprtost $\sigma$-algebre $A$ na $\Omega$ za:
            \begin{enumerate}
                \item $\cap$: $A \cap B = A \cap B \cap \Omega \cap \Omega \cdots \in \A$
                \item $\cup$: $A \cup B = A \cup B \cup \emptyset \cup \emptyset \cdots \in \A$
                \item $\backslash$: $A \backslash B = A  \cap \left( \Omega \backslash B \right) \in \A$
            \end{enumerate}

        \end{proof}

        \subsubsection{Mere}
            \begin{definicija}(Mera)
                Naj bo $\left( \Omega, \F \right)$ merljiv prostor in $\mu: \F \rightarrow [0, \infty] $. \\
                $\mu$ je mera na $\left( \Omega, \F \right)$ natanko tedaj ko: 
                \begin{enumerate}
                    \item $\mu(\emptyset) = 0$
                    \item $\mu$ je stevno aditivna: za $\forall$ zaporedje $(A_n)_{n \in \N} \subset \F$ paroma disjunktnih mnozic je $\mu(\cup_{n \in \N} A_n) = \sum_{n \in \N} \mu(A_n)$
                \end{enumerate}
            \end{definicija}

            Za mero $\mu$ na $\left( \Omega, F \right)$ recemo da je: 
            \begin{enumerate}
                \item koncna $\iff \mu(\Omega) < \infty$
                \item verjetnostna mera $\iff \mu(\Omega) = 1$
                \item $\sigma$-koncna $\iff \exists (A_n)_{n \in \N} \subset \F: \Omega = \cup_{n \in \N} A_n \ \text{in} \ \mu(A_n) < \infty \ \forall n \in \N$
            \end{enumerate}

            \begin{definicija}(Prostor z mero)
                $\left( \Omega, F, \mu \right)$ je prostor z mero $\iff \mu$ je mera na $(\Omega, F)$
            \end{definicija}

            \begin{definicija}
                Ce je $\left( \Omega, F, \mu \right)$ merljiv prostor. Potem za $A \in \F$ recemo:
                \begin{enumerate}
                    \item $A$ je $\mu$-zanemarljiva $\iff \mu(A) = 0$
                    \item $A$ je $\mu$-trivialna $\iff \mu(A) = 0$ ali $\mu(\Omega\backslash A) = 0$
                \end{enumerate}
            \end{definicija}

            Ce imamo neko lastnost $P(\omega)$ v $\omega \in A$, potem: 
            \begin{enumerate}
                \item $P(\omega)$ drzi $\mu$-skoraj povsod v $\omega \in A \iff A_{\neg P} := \{\omega \in A: \neg P(\omega)\}$ je $\mu$-zanemarljiva.       
                \item $P(\omega)$ drzi $\mu$-skoraj gotovo v $\omega \in A \iff A_{\neg P} := \{\omega \in A: \neg P(\omega)\}$ je $\mu$-zanemarljiva in $\mu$ je verjetnost.
                \item $P$ drzi $\mu$-s.p. na $A \iff P(\omega)$ drzi $\mu$-s.p. v $\omega \in A$. 
                \item $P$ drzi $\mu$-s.g. na $A \iff P(\omega)$ drzi $\mu$-s.g. v $\omega \in A$. 
            \end{enumerate}
            \begin{zgled}
                Nicelna mera na $\F$ (torej preslikava $\mu(A) \rightarrow 0, \ \forall A \in \F)$ je vedno mera na katerikoli $\sigma$-algebri.
            \end{zgled}      
            
            \begin{zgled}
                Ce definiramo $c_\Omega:2^\Omega \rightarrow [0, \infty]$ kot $c_\Omega(A) := |A|$ ce je $A$ koncna podmnozica $\Omega$ in $c_\Omega(A) := \infty$ ca je neskoncna podmnozica $\Omega$, je $c_\Omega$ tako imenovana stevna mera na $\Omega$. Ko je $\Omega$ koncna in neprazna, potem je $\frac{c_\Omega}{|\Omega|}$ verjentostna mera na $\Omega$.  
            \end{zgled}

            \begin{zgled}
                Ce definiramo $\delta_x:2^\Omega \rightarrow [0, \infty]$ za fiksen $x \in \Omega$, tako da za $A \in 2^\Omega, \delta_x(A):=0$ ce $x \notin A$ in $\delta_x(A):=1$ ce $x \in A$, potem je $\delta_x$ tako imenovana Diracova mera za $x$. Katerakoli podmnozica $\Omega \backslash \{x\}$ je $\delta_x$-zanemarljiva.                
            \end{zgled}
            
            \begin{trditev}(Lastnosti mere)
                Naj bo $\mu$ mera na merljivem prostoru $\left( \Omega, \F \right)$. Potem velja naslednje:
                \begin{enumerate}
                    \item $\mu$ je aditivna: $\mu(A \cup B) = \mu(A) + \mu(B)$ kadarkoli $A \cap B = \emptyset$ in $\{A, B\} \in \F$.
                    \item $\mu$ je monotona: $A \subset B$ in $ \{A, B\} \in \F$ $ \Rightarrow \mu(A) \leq \mu(B)$.
                    \item $\mu$ je zvezna od spodaj: $\mu(\cup_{n \in \N}A_n) = \uparrow\text{-}\lim_{n \rightarrow \infty}\mu(A_n)$ kadarkoli je $(A_n)_{n \in \N}$ narascajoce zaporedje v $\F$.
                    \item $\mu$ je stevno subaditivna: $\mu(\cup_{n \in \N}A_n) \leq \sum_{n \in \N}\mu(A_n)$ kadarkoli je $(A_n)_{n \in \N}$ zaporedje v $\F$.
                    \item Predpostavimo da je $\mu$ koncna. $\mu(\Omega\backslash A) = \mu(\Omega) - \mu(A)$ za vse $A \in \F$. Se vec, $\mu$ je zvezna od zgoraj: $\mu(\cap_{n \in \N}A_n) = \downarrow\text{-}\lim_{n \rightarrow \infty}\mu(A_n)$ kadarkoli je $(A_n)_{n \in \N}$ padajoce zaporedje v $\F$.
                    \item Za $A \in \F$ je $\F|_A:=\{B \cap A: B \in \F\}$; potem je $\mu_A:= \mu_{\F|_A}$ mera na $\F|_A$. Imenuje se restrikcija/skrcitev mere $\mu$ na $A$.  
                \end{enumerate}             
            \end{trditev}

            \begin{proof}
                \begin{enumerate}
                    \item $\left(A, B, \emptyset, \emptyset, \cdots \right)$ je zaporedje medseboj disjunktnih mnozic v $\F$, torej po stevni aditivnosti velja $\mu(A \cup B \cup \emptyset \cup \cdots) = \mu(A) + \mu(B) + \mu(\emptyset) + \cdots = \mu(A) + \mu(B)$.
                    \item $B = A \cup (B \backslash A)$ in uporabimo koncno aditivnost (1.).
                    \item $\left(A_1, A_2\backslash A_1, A_3\backslash A_2, \cdots \right)$ je zaporedje paroma disjunktnih mnozic v $\F$ z unijo $\cup_{n \in \N}A_n$. Uporabimo stevno aditivnost in za tem se koncno aditivnost.
                    \item $\left(A_1, A_2\backslash A_1, A_3\backslash (A_1 \cup A_2), \cdots \right)$ je zaporedje paroma disjunktnih mnozic v $\F$ z unijo $\cup_{n \in \N}A_n$. Uporabimo stevno aditivnost in za tem se monotonost (2.).
                    \item Prvi del sledi iz koncne aditivnosti (1.). Za mnozici vzamemo $A \in \Omega$ in  $B = \Omega \backslash A$. Dobimo $\mu(\Omega) = \mu(A) + \mu(\Omega \backslash A)$. Drugi del sledi iz zveznosti od spodaj (3.) tako da jo uporabimo na $\left(\Omega\backslash A_n \right)_{n \in \N}$. $\mu(\Omega) - \mu(\cap_{n \in \N}A_n) = \mu\left( \cup_{n \in \N}\Omega\backslash A_n\right) = \lim_{n \rightarrow \infty}\mu(\Omega\backslash A_n) = $ $\lim_{n \rightarrow \infty}\mu(\Omega) - \mu(A_n) = \mu(\Omega) - \lim_{n \rightarrow \infty}\mu(A_n)$ $ \Rightarrow \mu(\cap_{n \in \N}A_n) = \lim_{n \rightarrow \infty}\mu(A_n).$
                    \item Preveriti moramo, da je $\F|_A$ $\sigma$-algebra na $A$. Kasneje bomo videli da je $\F|_A = 2^A \cap \F$ in bo dokaz sledil iz tega.
                \end{enumerate}
            \end{proof}

            \begin{zgled}(Borel-Cantellijeva lema)
                Naj bo $\left( \Omega, \F, \mu \right)$ merljiv prostor in $(A_n)_{n \in \N}$ zaporedje v $\F$, da velja $\sum_{n \in \N}\mu(A_n) < \infty$. Potem je $\mu(\limsup_{n \rightarrow \infty}A_n) = 0$.
            \end{zgled}

            \begin{proof}
                
            \end{proof}

            \begin{zgled}
                Ce je $P$ vejretnost na $\left( \Omega, \F \right)$, potem je $P^{-1}(\{0, 1\})$ pod-$\sigma$-algebra na $\F$. Tako imenovana $P$-trivialna $\sigma$-algebra.
            \end{zgled}

        \subsubsection{Merjlive preslikave in generirane $\sigma$-algebre}
            
            \begin{definicija}(Generirana $\sigma$-algebra)
                Naj bo $\A \subset 2^\Omega$; potem $$\sigma_\Omega(\A):=\cap\{\F \in 2^{2^\Omega}: \F \ \text{je} \ \sigma\text{-algebra na} \ \Omega \ \text{in} \ \A \subset \F \}$$ imenujemo $\sigma$-algebra generirana na $\Omega$ z $\A$. Je najmanjsa $\sigma$-algebra, ki vkljucuje druzino podmnozic $\A$.
            \end{definicija}

            \begin{opomba}
                $2^\Omega$ je gotovo $\sigma$-algebra na $\Omega$, ki vsebuje $\A$, torej je $\sigma_\Omega(\A)$ neprazna.
            \end{opomba}

            Za dve $\sigma$-algebri $\mathcal{B}_1$ in $\mathcal{B}_2$ na $\Omega$ je mnozica $\mathcal{B}_1 \vee \mathcal{B}_2 := \sigma_\Omega\left(\mathcal{B}_1 \cup \mathcal{B}_2\right)$ skupek $\mathcal{B}_1$ in $\mathcal{B}_2$. Bolj splosno
            za druzino $\left(\mathcal{B}_\lambda\right)_{\lambda \in \Lambda} \sigma$-algeber na $\Omega$ pravimo, da je $\vee_{\lambda \in \Lambda}\mathcal{B}_\lambda := \sigma_\Omega\left(\cup_{\lambda \in \Lambda}\mathcal{B}_\lambda\right)$ njen skupek.

            \begin{opomba}
                Razlog zakaj so generirane $\sigma$-algebre pomembne v teoriji mere je, ker le redko lahko eksplicitno
                podamo vse elemente $\sigma$-algebre, ki bi jo zeleli, ampak pogosto lahko eksplicitno podamo njene generatorje.  
            \end{opomba}

            \begin{definicija}(Zacetna in koncna struktura)
                Naj bo $f:\Omega \rightarrow \Omega'$. Za podano $\sigma$-algebro $\F'$ na $\Omega'$  definiramo $$\sigma^{\F'}(f):= f^{-1}(\F'):=\{f^{-1}(A'):A' \in \F'\},$$
                zacetno strukturo za $f$ glede na $\F'$. (oziroma $\sigma$-algebra generirana z $f$ glede na $\F'$).

                Za podano $\sigma$-algebro $\F$ na $\Omega$ definiramo $$\sigma^{\Omega'}_{\F}(f):=\{A' \in 2^{\Omega'}: f^{-1}(A') \in \F\},$$
                koncno strukturo za $f$ na $\Omega'$ glede na $\F$.
            \end{definicija}

            \begin{definicija}
                Za dano $\sigma$-algebro $\F'$ na $\Omega'$ in $\sigma$-algebro $\F$ na $\Omega$ pravimo da je $f$ $\F/\F'$-merljiva preslikava $\iff f^{-1}(A')\in \F$ za vse $A' \in \F'$.
            \end{definicija}

            \begin{opomba}
                \begin{enumerate}
                    \item V notacijah $\sigma_\Omega(\A)$ in $\sigma^{\F'}(f)$ spuscamo $\Omega$ in $\F'$ kadar je to jasno iz konteksta. Torej pisemo preprosto $\sigma(\A)$ in $\sigma(f)$.
                    \item V primeru ko je zaloga vrednosti $f$ stevna in nimamo podane ne $\F'$ ali $\Omega'$ za $\Omega'$ vazamemo $Z_f$ in za $\F'$ vazamemo $2^{\Omega'}$.
                    \item Izmed objektov, ki smo ju uvedli v definiciji 1.21 je zacetna struktura bolj sugestivna/pomembna.
                \end{enumerate}
            \end{opomba}

            \begin{definicija}
                Za dano $\sigma$-algebro $\F$ na $\Omega$ in $\F'$ na $\Omega'$ definiramo $$\F/\F':= \{g \in \Omega'^\Omega: g \ \text{je} \ \F/\F'\text{-merljiva}\}.$$
            \end{definicija}

            \begin{zgled}
                Konstantna funkcija je vedno merljiva, ne glede na $\sigma$-algebro. Za poljubno $\sigma$-algebro $\F$ na $\Omega$ je $id_\Omega \in \F/\F'$.
            \end{zgled}

            \begin{definicija}(Indikator)
                Za $A \subset \Omega$ definiramo $\mathds{1}_{A_\Omega}:\Omega\rightarrow\{0, 1\}$:
                $$\mathds{1}_{A_\Omega}(x) := 
                    \begin{cases}
                        1, & \ x \in A \\
                        0, & \text{sicer}
                    \end{cases}$$
                Funkciji pravimo indikator mnozice $A$ v $\Omega$. Pisali bomo $\mathds{1}_A$ in predvidevali da se $\Omega$ da razbrati iz konteksta.
            \end{definicija}

            \begin{zgled}
                Naj bo $A \subset \Omega$. Potem $\sigma^{2^{\{0, 1\}}}(\mathds{1}_A) = \sigma_\Omega (A) = \{\emptyset, \Omega, A, \Omega\backslash A\}$. Ce je nadaljno $\F$ $\sigma$-algebra na $\Omega$ potem je  $\mathds{1}_A \in \F/2^{\{0, 1\}} \iff A \in \F$ 
            \end{zgled}

            \begin{trditev}
                Naj bodo $\F, \mathcal{G}, \mathcal{H}$ $\sigma$-algebre (vsaka na svoji mnozici). Naj bo $f \in \F/\mathcal{G}$ in $g \in \mathcal{G}/\mathcal{H}$. Potem je $g \circ f \in \F/\mathcal{H}.$ Z besedami to pomeni: kompozitumi merljivih preslikav so merljive preslikave.
            \end{trditev}

            \begin{proof}
                $(g \circ f)^{-1}(H) = f^{-1}(g^{-1}(H))$ za $\forall H$ $\in \mathcal{H}$
            \end{proof}

            \begin{trditev} (Lastnosti preslikav)
                Naj bo $f: \Omega \rightarrow \Omega'$.
                \begin{enumerate}
                    \item Naj bo $\F'$ $\sigma$-algebra na $\Omega'$. $\sigma^{\F'}(f)$ je $\sigma$-algebra na $\Omega$; je najmanjsa $\sigma$-algebra $\mathcal{F}$ na $\Omega$ da velja $f \in \mathcal{F}/\F'$.
                    \item Naj bo $\F$ $\sigma$-algebra na $\Omega$. $\sigma_{\F}^{\Omega'}(f)$ je $\sigma$-algebra na $\Omega'$; je najvecja $\sigma$-algebra $\mathcal{F}'$ na $\Omega'$ da velja $f \in \mathcal{F}/\F'$.
                    \item Naj bo $\F'$ $\sigma$-algebra na $\Omega'$ in $\F$ $\sigma$-algebra na $\Omega$, potem $f \in \F/\F' \iff \sigma^{\F'}(f) \subset \F \iff \sigma_{\F}^{\Omega'}(f) \supset \F'.$
                    \item Naj bo $\A' \subset 2^{\Omega'}$. $\sigma_{\Omega'}(\A')$ je najmanjsa $\sigma$-algebra na $\Omega'$, ki ima $\A'$ za svojo podmnozico. Naj bo $\F$ $\sigma$-algebra na $\Omega$, potem je $f \in \F/\sigma_{\Omega'}(\A') \iff (f^{-1}(A') \in \F$ za vse $A' \in \A'$). Natancno zapisemo $\sigma^{\sigma_{\Omega'}(\A')}(f) = \sigma_\Omega(\{f^{-1}(A'):A' \in \A'\})$. V posebnem je $f^{-1}(\sigma_{\Omega '}(\A ')) = \sigma_{\Omega}(f^{-1}(\A').$
                \end{enumerate}
            \end{trditev}

            \begin{proof}
                \begin{enumerate}
                    \item $f^{-1}(\F')$ je $\sigma$-algebra na $\Omega$: $\forall \emptyset : f^{-1}(\emptyset) \in f^{-1}(\F');$ za $A'\in \F'$ je $\Omega \backslash f^{-1}(A') = f^{-1}(\Omega \backslash A') \in f^{-1}{\F'};$ za zaporedje $(A_i)_{i \in \N}$ je $\cup_{i \in \N}f^{-1}(A_i') = f^{-1}(\cup_{i \in \N}A_i') \in f^{-1}(\F').$ Drugi del je jasen
                    \item Podoben dokaz kot 1.
                    \item Je direktna posledica definicije zacetne in koncne strukture.
                    \item Prvi del: je jasen iz defnicije generirane $\sigma$-alebre. \\
                    Drugi del: $(\Rightarrow):$ je jasen iz definicije generirane $\sigma$-algebre. \\
                    $(\Leftarrow):$ $A' \subset \sigma_{\F}^{\Omega '}(f)$ torej sledi $\sigma_{\Omega'}(A')\subset \sigma_{\F}^{\Omega '}(f)$ in zato sledi $f \in \F\backslash\sigma_{\Omega'}(A'). $ \\
                    Tretji del: V drugem delu vzamemo $\F = \sigma_{\Omega}(f^{-1}(A'))$, kar nam da $f \in \sigma_{\Omega}(f^{-1}(A'))\backslash\sigma_{\Omega'}(A')$. Opazimo, da je $f^{-1}(A')\subset f^{-1}(\sigma_{\Omega'}(A'))$ in zato po tocki 1. velja $\sigma_{\Omega}(f^{-1}(A')) \subset f^{-1}(\sigma_{\Omega'}(A'))$.
                \end{enumerate}
            \end{proof}

            \begin{opomba}
                Tocka 4 nam pove da je dovolj da dokazemo  merljivo lastnost na mnozici generatorjev. Se en nacin zapisa $f \in \F/\sigma_{\Omega'}(\A') \iff (f^{-1}(A') \in \F$ za vse $A' \in \A'$). Natancno zapisemo $\sigma^{\sigma_{\Omega'}(\A')}(f) = \sigma_\Omega(\{f^{-1}(A'):A' \in \A'\})$ je $f^{-1}(\sigma_{\Omega'}(\A')) = \sigma_\Omega(f^{-1}(\A'))$, kar bomo interpretirali kot operacija dobivanja praslik in generiranih $\sigma$-algeber komutirati.
            \end{opomba}

            \begin{definicija}
                Pisemo $\A|_A := \{A'\cap A : A'\in \A\}$ za sled $\A$ na $A$.
            \end{definicija}

            \begin{opomba}
                Ce je $\F$ zaprta za $\cap$ in $A \in \F$, potem je $\F|_A = \F \cap 2^A$.
            \end{opomba}

            \begin{posledica}
                Naj bo $\A \subset 2^\Omega$. Ce je $A \subset \Omega$, potem $$ \sigma_\Omega(\A)|_A = \sigma_A(\A|_A);$$ v primeru ce je $\A$ $\sigma$-algebra na $\Omega$, potem je $\A|_A$ $\sigma$-algebra na $A$.
            \end{posledica}

            \begin{proof}
                Po prejsnji trditvi (1. tocka) je $\sigma_\Omega(\A)|_A = \sigma^{\sigma_\Omega(\A)}(id_A)$ $\sigma$-algebra na $A$ ki vsebuje $\A|_A$, torej velja $\sigma_A(\A|_A) \subset \sigma_\Omega(\A|_A).$ Po 2. tocki je $\mathcal{C} := \{ C \in 2^\Omega: C \cap A \in \sigma_A(\A|_A)\} = \sigma_{\sigma_A(\A|_A)}^\Omega(id_A)$ $\sigma$-algebra na $\Omega$, ki vsebuje $\A$, torej $\sigma_\Omega(\A) \subset \mathcal{C}$, torej $\sigma_\Omega(\A)|_A \subset \sigma_A(\A|_A).$

            \end{proof}


            \begin{opomba}
                Kako lahko v sposnem dolocimo $\sigma_\Omega(\A)$? Zacnemo z $\A$ karkoli kar mora biti v $\sigma_\Omega(\A)$ da zadosca pogojem $\sigma$-algebre, vse komplemente, stevne unije, $\emptyset$, $\Omega$ in stevne unije teh itd. Po tem postopku dokazemo, da je to kar imamo $\sigma$-algebra.
            \end{opomba}

            \begin{zgled}
                Naj bosta $\{E, F\} \subset 2^\Omega.$ Potem mora $\sigma_\Omega(\{E, F\})$ vsebovati $\{\emptyset, E, F, E\backslash F, E \cap F, \cdots\}$ (Particije na $\Omega$ inducirane z $E, F$). Torej $\sigma_\Omega(\{E, F\}) \supset \sigma_{\Omega}(\mathcal{P})$. Ampak $\sigma_{\Omega}(\mathcal{P})$ je $\sigma$-algebra na $\Omega$, ki vsebuje $\{E, F\}$, torej $\sigma_\Omega(\{E, F\}) = \sigma_{\Omega}(\mathcal{P}) = \sigma\mathcal{P}$ iz zgleda 1.8.
            \end{zgled}

            \begin{trditev}
                Naj bo $f:\Omega \rightarrow \Omega'$ in naj bo $\F$ $\sigma$-algebra na $\Omega$ ter $\F'$ $\sigma$-algebra na $\Omega'$.
                \begin{enumerate}
                    \item Ce je $A' \subset \Omega$ taksna, da $f:\Omega \rightarrow A'$, potem je $f \in \F/\F'$ natanko tedaj ko je $f \in \F/(\F'|_{A'})$.
                    \item Za $A \subset \Omega$, ce je $f \in \F/\F'$, je $f|_{A} \in (\F|_A)/\F'$.
                    \item Ce za zaporedje $(A_i)_{i \in \N}$ v $\F$, az katerega je $\Omega = \cup_{i \in \N}A_i$, velja $f|A_{i} \in (\F|_{A_{i}})/\F'$ za vsak $i \in \N$, potem $f \in \F/\F'$.
                \end{enumerate}
            \end{trditev}

            \begin{proof}
                \begin{enumerate}
                    \item Za $H'\in \F'$ je $f^{-1}(H') = f^{-1}(H'\cap A').$
                    \item Za $F' \in \F'$ je $(f|_A)^{-1}(F') = A \cap f^{-1}(F') \in \F|_A$.
                    \item Za $F'\in \F'$ je $f^{-1}(F') = \cup_{i \in \N}f^{-1}(F')\cap A_i \underbrace{=}_{\cup_{i \in \N}A_i = \Omega} \cup_{i \in \N}\underbrace{(f|_{A_i})^{-1}(F')}_{\in \F|_{A_i} = \F\cap 2^{A_i} \subset \F}$
                \end{enumerate}
            \end{proof}

            \begin{opomba}
                Tocki 1. in 2. pomenita da se merljivost obnasa lepo pod omejitvami. Tocka 3. pa nam pove da je lahko merljivost preverjena "lokalno". 
            \end{opomba}

        \subsubsection{Borelove mnozice na razsirjeni realni osi in Borelova merljivost numericnih funkcij}   
            \begin{definicija}
                Naj bo $[-\infty, \infty] := \R \cup \{-\infty, \infty\}$ razsirjena realna os, opremljena z naravno relacijo $\leq$. Vpeljemo druzino mnozic $\mathcal{B}_{[-\infty, \infty]}:= \sigma_{[-\infty, \infty]}(\{[-\infty, a]:a \in \R\})$, ki ji pravimo Borelova $\sigma$-algebra na $[-\infty, \infty]$. Za $A \subset [-\infty, \infty]$ vpeljemo druzino mnozic $\mathcal{B}_A = \mathcal{B}_{[-\infty, \infty]}|_A$, ki ji pravimo Borelova $\sigma$-algebra na $A$. 
            \end{definicija}

            \begin{opomba}
                Funkcije, ki so merljive glede na $\mathcal{B}_{[-\infty, \infty]}$ na kodomeni, so nekako natanko tiste, ki se "lepo" 
                obnasajo s stalisca integracije. Pričakovanje tega, kar sledi, nam je všeč tudi zato, ker na $(\R, \mathcal{B}_\R)$ 
                lahko definiramo prijetno - netrivialno translacijsko invariantno - tako imenovano Lebesquovo mero. Prav tako lahko 
                trdimo, da je reči, da je numerična preslikava $f$ merljiva (z vidika merjenja zanimiva),
                 treba vsaj izmeriti množice $\{f \leq a\}:= f^{-1}{[-\infty, a]}$ za vsak $a \in \R$ (zlasti, 
                ob malce predvidevanju vsebine drugega dela teh zapisov, za naključno spremenljivko $X$ bi 
                želeli biti sposobni povedati, kaj je verjetnost dogodkov $\{X \leq a\}$ za $a \in \R$).
            \end{opomba}

            \begin{zgled}
                Vsi intervali in stevne podmnozice $[-\infty, \infty]$ pripadajo $\mathcal{B}_{[-\infty, \infty]}$. Prav tako vse zaprte in odprte podmnozice $[-\infty, \infty]$ pripadajo $\mathcal{B}_{[-\infty, \infty]}$. Ce je $A \subset [-\infty, \infty]$ stevna, potem je $\mathcal{B}_A = 2^A.$
            \end{zgled}
        
            \begin{definicija}
                Ce $f$ slika v $[-\infty, \infty]$ (je numericna), potem: 
                \begin{enumerate}
                    \item $\sigma(f) = \sigma^{\mathcal{B}_{[-\infty, \infty]}}(f)$.
                    \item Za $\sigma$-algebro $\F$ na $D_f$ recemo da $f$ je  $\F$-merljiva $\iff f \in \F/\mathcal{B}_{[-\infty, \infty]}$.
                    \item Za $g:D_f \rightarrow [-\infty, \infty]; g \land f :=$ min$\{g, f\}$,  $g \lor f:=$ max$\{g, f\}$, $f^+:=$ max$\{f, 0\}$ in  $f^-:=$ max$\{-f, 0\}$.
                \end{enumerate}
            \end{definicija}

            \begin{zgled}
                    $\mathcal{B}_\R = \sigma_\R(\{(-\infty, a]: a \in \R\})$. Posledicno po trditvah 1.29 in 1.39 za $\sigma$-algebro $\F$ na $\Omega$ in $f:\Omega \rightarrow \R$, $f$ Borelovo merljiva $\iff$ $f \in \F/\mathcal{B}_\R \iff \{f \leq a\}:= f^{-1}((-\infty, a]) \in \F$ za $\forall a \in \R.$
            \end{zgled}

            \begin{definicija} (Aritmetika z $\infty$)
                 \begin{enumerate}
                    \item $0\cdot (\pm\infty) := 0$
                    \item $\infty + (-\infty) := 0$
                 \end{enumerate}
                 Preostanek aritmetike v $[-\infty, \infty]$ vpeljemo naravno, npr. $a \cdot \infty = sgn(a)\infty$ za $a \in [-\infty, \infty]\backslash\{0\}$, $a + \infty = \infty$ itd.
            \end{definicija}

            \begin{trditev}
                Za $A \subset [-\infty, \infty]$ in $f:A \rightarrow [-\infty, \infty]$ zvezna ter $f \in \mathcal{B}_A\backslash\mathcal{B}_{[-\infty, \infty]}.$ Ce je $\F$ $\sigma$-algebra 
                in $\{f, g\} \subset \F\backslash\mathcal{B}_{[-\infty, \infty]},$ potem je $\{f+g, fg\} \subset \F\backslash\mathcal{B}_{[-\infty, \infty]}$ in $\{\{f=g\}, \{f\subset g\}, \{f \leq g\}\} \subset \F$.
            \end{trditev}

            \begin{proof}
                Brez dokaza.
            \end{proof}

            \begin{trditev}
                Naj bo $\F$ $\sigma$-algebra in $\left(f_n\right)_{n \in \N}$ zaporedje v $\F\backslash\mathcal{B}_{[-\infty, \infty]}.$
                Potem je $$\{\sup_{n \in \N}f_n, \inf_{n \in \N}f_n, \limsup_{n \rightarrow \infty}f_n, \liminf_{n \rightarrow \infty}f_n \} \subset \F\backslash\mathcal{B}_{[-\infty, \infty]}.$$
                Ce je $f_n \geq 0$ za $n \in \N$, je $\sum_{n \in \N}f_n \in \F\backslash\mathcal{B}_{[0, \infty]}.$
            \end{trditev}

            \begin{proof}
                $\mathcal{B}_{[-\infty, \infty]} = \sigma_{[-\infty, \infty]}(\{[-\infty, a]: a\in\R\})$. Za $a \in \R$ 
                $\left(sup_{n \in \N}f_n\right)^{-1}\left([-\infty, a]\right) = \{sup_{n \in \N}f_n \leq a\} = \{f_n \leq a \ \forall n \in \N\} =
                 \cap_{n \in \N}{f_n \leq a}= \cap_{n \in \N}\underbrace{f_n^{-1}([-\infty, a])}_{\in \F \because f_n \in \F\backslash\mathcal{B}_{[-\infty, \infty]}}$
                 Da je $inf_{n \in \N}f_n \in \F\backslash\mathcal{B}_{[-\infty, \infty]}$ utemeljimo tako, da zapisemo $inf_{n \in \N} = -sup_{n \in \N} - f_n$ in opazimo, da je $-id_{[-\infty, \infty]} \in \mathcal{B}_{[-\infty, \infty]}\backslash\mathcal{B}_{[-\infty, \infty]}$
                 limsup in liminf sta kombinaciji sup in inf. Koncno v primeru, da je $f_n \geq 0$ $\forall n \in \N$ je $\sum_{n\in\N}f_n = \lim_{n \rightarrow \infty}\sum_{k = 1}^{n}f_k = limsup_{n \rightarrow \infty}\underbrace{\sum_{k = 1}^nf_k}_{\in \F\backslash\mathcal{B}_{[-\infty, \infty]} \text{po trditvi 1.44}}$
                 Kar nam da $\sum_{n \in \N}f_n \in \F\backslash\mathcal{B}_{[0, \infty]}$.
            \end{proof}

            \begin{posledica}
                Naj bo $\F$ $\sigma$-algebra, potem je $\{\max(f, g), \min(f, g), f^+, f^-, |f|\} \subset \F\backslash\mathcal{B}_{[-\infty, \infty]}$ za $\{f, g\} \subset \F\backslash\mathcal{B}_{[-\infty, \infty]}.$ Poleg tega je 
                $\{\{f_n \ \text{konvergira, ko gre} \ n \rightarrow \infty\},$ 
                $ \{f_n \text{konvergira k vrednosti iz} \ \R \ \text{ko gre} \ n \rightarrow \infty\}, \{\lim_{n \rightarrow \infty}f_n = f_0\}\} \subset \F$ za vsako
                zaporedje $(f_n)_{n \in \N_0}$ v $\F\backslash\mathcal{B}_{[-\infty, \infty]}.$
            \end{posledica}

            \begin{proof}
                Brez dokaza.
            \end{proof}

        \subsubsection{Argumenti monotonega razreda}
            IDEJA: Zelimo dokazati tridtev, ki se tice vseh funkcij iz $\F\backslash\mathcal{B}_{[-\infty, \infty]}.$ Najprej pokazemo trditev za 
            $\mathds{1}_A, A \in \F$. $\rightarrow$ izrek o monotonem razredu $\rightarrow$ trditev velja v splosnem.

            \begin{definicija}
                Naj bo $\F$ $\sigma$-algebra na $\Omega$ in $f:\Omega\rightarrow [0, \infty]$.
                f je $\F$-enostavna $\iff$ $f \in \F\backslash\mathcal{B}_{[-\infty, \infty]}$ in $\mathcal{Z}_f$ je koncna mnozica.
            \end{definicija}

            \begin{trditev}
                Naj bo $(\Omega, \F)$ merljiv prostor in $f:\Omega\rightarrow [0, \infty]$. $f$ je $\F$-enostavna $\iff$
                $f = \sum_{i = 1}^{n}c_i\mathds{1}_{A_i}$ za neke $c_i \in [0, \infty), A_i \in \F, i \in [n]$ za nek $n \in \N$. Naprej, ce je $f \in \F\backslash\mathcal{B}_{[-\infty, \infty]},$
                potem je $min\left(2^{-n}[2^nf], n\right)_{n \in \N}$ zaporedje $\F$-enostavnih funkcij, ki narascajo proti ($\uparrow$) $f$. (Celo enakomerno na vsaki mnozici na kateri je $f$ omejena).
            \end{trditev}

            \begin{proof}
                $(\Rightarrow):$ $f = \sum_{\mathcal{Z}\backslash\{0\}}a\cdot\mathds{1}_{\{f = a\}}, \{f = a\}$ pomeni $f^{-1}(\{a\}) \in \F.$
                $(\Leftarrow):$ Baje da je ocitno
                Drugi del je jasen.
            \end{proof}

            \begin{opomba}
                \begin{enumerate}
                    \item $f \in \F\backslash\mathcal{B}_{[-\infty, \infty]} \iff f = 1 -$ $\F$-enostavnih funkcij.
                    \item $f \in \F\backslash\mathcal{B}_{[-\infty, \infty]} \iff f =$ limita linearnih kombinacij indikatorjev mnozic iz $\F$.
                \end{enumerate}

            \end{opomba}

            \begin{posledica}(izrek o monotonem razredu)
                Naj bo $\F$ $\sigma$-algebra in $\mathcal{M} \subset \F\backslash\mathcal{B}_{[0, \infty]}.$ Ce velja
                \begin{enumerate}
                    \item $\mathds{1}_A \in \mathcal{M} \ \forall A \in \F$
                    \item $\mathcal{M}$ je konveksen stozec; tj. $af + g \in \mathcal{M} \ \forall a \in [0, \infty] \ \forall f \in \mathcal{M} \ \forall g \in \mathcal{M}$
                    \item $\mathcal{M}$ zaprt za $\uparrow$ limite; tj. $\lim_{n \rightarrow \infty}f_n \in \mathcal{M} \ \forall$ zaporedje $(f_n)_{n \in \N} $ v $\mathcal{M}$
                \end{enumerate}
                Potem je $\mathcal{M} =  \F\backslash\mathcal{B}_{[-\infty, \infty]}.$
            \end{posledica}

            \begin{proof}
                Po 1. in 2. $\mathcal{M}$ vsebuje vse $\F$-enostavne funkcije. Vsaka funckija iz $\F\backslash\mathcal{B}_{[-\infty, \infty]}$ je po trditvi 1.41 
                $\uparrow$-limita $\F$-enostvnih funkcij in zato pripada $\mathcal{M}$ po 3.
            \end{proof}

            \begin{trditev}
                (Doob-Dynkinova faktorizacijska lema)
                Naj bo $X:\Omega \rightarrow A$ in  $(A, \A)$ merljiv prostor. Potem je 
                $Y \in X^{-1}(\A)\backslash\mathcal{B}_{[-\infty, \infty]} \iff \left(\exists h \in \A\backslash\mathcal{B}_{[-\infty, \infty]}, da je Y = \underbrace{h \circ X}_{h(X)}\right).$  
            \end{trditev}

            \begin{proof}
                $(\Leftarrow):$ $X \in X^{-1}(\A)\backslash\A$ in $h \in \A\backslash\mathcal{B}_{[-\infty, \infty]} \Rightarrow$ (po trditvi 1.21) $h \circ X \in X^{-1}(\A)\backslash\mathcal{B}_{[-\infty, \infty]}.$

                $(\Rightarrow):$ Dopisi
            \end{proof}

            \begin{posledica}
                $\mathcal{M} = X^{-1}(\A)\backslash\mathcal{B}_{[0, \infty]}$
            \end{posledica}

            Za splosen $Y \in X^{-1}(\A)\backslash\mathcal{B}_{[-\infty, \infty]}$ je $\{Y^+, Y^-\} \subset X^{-1}(\A)\backslash\mathcal{B}_{[0, \infty]},$
            zato po pravkar dokazanem obstoju $h_+, h_-$ iz $X^{-1}(\A)\backslash\mathcal{B}_{[0, \infty]}$ sledi, da je \dots

            \begin{definicija}
                Naj bo $D \subset 2^\Omega.$ $D$ je Dynkinov (tudi $\lambda$-) sistem na $\Omega \iff \Omega \in D$ in $(B\backslash A \in D \text{za} \ D \ni A \subset B \in D)$ in
                $(\cup_{i \in \N}A_i \in D \forall \uparrow \text{zaporedje} (A_i)_{i \in \N} v D).$ $D$ je $\pi$-sistem $\iff D$ je zaprta za $cap$.

            \end{definicija}

            \begin{zgled}
                $\{(-\infty, a]: a \in \R\}$ je $\pi$-sistem.
            \end{zgled}

            \begin{trditev}
                Naj bo $D \subset 2^{\Omega}.$ $D$ je Dynkinov sistem na $\Omega \iff \Omega \in D$, $D$ je zaprt za $c^\Omega$, 
                $\cup_{i \in \N}A_i \in D \forall$ zaporedje $(A_i)_{i \in \N}$ v $D$ ki ima $A_i \cap A_j = \emptyset$ za $i \neq j$ iz $\N$.

            \end{trditev}


            ... dopolni


        \subsubsection{Lebesque-Stieltsove mere}
            \begin{izrek}
                (Lebesque-Stieltsove mere)
            \end{izrek}

            \begin{proof}
                pisi
            \end{proof}

            \begin{definicija}
                Meri $\mu$ iz izreka pravimo Lebesque-Stieltsova mera prirejena $F$ in jo oznacimo z $dF$.
                ($\mathcal{L}:= d(id_\R)$ je Lebesquova mera na $\R$, ne obstaja razsiritev $\mathcal{F}$ na mero na $(\R, 2^\R)$).
            \end{definicija}

            \begin{trditev}
                Naj bo $F:\R \rightarrow \R$ nepadajoca in zvezna z desne. Mera $dF$ je: $\sigma$-koncna; koncna $\iff$ $F$ je omejena
                verjetnostna $\iff$ $\lim_{\infty}F - \lim_{-\infty}F = 1$ Za $x \in \R$ je $dF(\{x\}) = F(x) - \underbrace{F(x-)}_{\text{leva limita $F$ v $x$}}$. 

            \end{trditev}
            \begin{proof}
                $\sigma$-koristnost: $\cup_{n \in \mathbb{Z}}(n, n+1]=\R$
                $dF((n, n+1]) = F(n + 1) - F(n) \leq \infty \forall n \in \mathbb{Z}$.
                ... poglej v skripto
                Od tod dobimo karakterizacijo koncnosti $dF$ in kdaj je $dF$ verjetnostna mera.

                Za $x \in \R$ je $(x - \frac{1}{n}, x] \downarrow \{x\}$, ko gre $n \rightarrow \infty$ cez $\N$ in zato je po zveznosti
                $dF$ od zgoraj na mnozicah s koncno mero $dF(\{x\}) = \lim_{n \rightarrow \infty}dF((x - \frac{1}{n}, x]) = \lim_{n \rightarrow \infty}F(x) - F(x - \frac{1}{n})$.
            \end{proof}

            \begin{zgled}
                skripta
            \end{zgled}

            \begin{zgled}
                skripta (Contorjeva mnozica)
            \end{zgled}

    \subsection{Integracija na merljivih prostorih}
        \begin{zgled}(nevem ce je zgled)
            Naj bo $\mathcal{P}$ particija $\Omega$ in $\mu':\mathcal{P} \rightarrow [0, \infty)$ in
            $f':\mathcal{P} \rightarrow [0, \infty)$ ($\mu$ je mera na $\sigma_\Omega(\mathcal{P})=\{\cup Q: Q \in 2^{\mathcal{P}}\}; \ 
            \mu(\cup Q):= \sum_{p \in Q}\mu'(p)$ za $Q \subset \mathcal{P}$), ($f:\Omega \rightarrow [0, \infty); f(\omega) = f'(p)$ za $\omega \in p \in \mathcal{P}$)
            Potem je $\sum_{p \in \mathcal{P}}f'(p)\cdot\mu'(p) = \sum_{r \in \mathcal{Z}_f}r\cdot\mu(\underbrace{\{f = r\}}_{f^{-1}(\{r\})})$.

            tukej sta se dve interpretaciji tega
        \end{zgled}

        $\sum_{p \in \mathcal{P}}f'(p)\mu'(p) = \int_a^bf(z)dz$ Riemann - Darbouxov integral odsekoma konstantne funckije 
        $f: [a, b] \rightarrow [0, \infty)$.

        \subsubsection{Lebesqueov integral}
            \begin{definicija}
                Naj bo $(\Sigma, \F, \mu)$ prostor z mero in $f \in \F\backslash\mathcal{B}_{[-\infty, \infty]}$
                \begin{itemize}
                    \item Ce je $f$ $\F$-enostvna, potem je
                    $$
                        \int fd\mu := \sum_{a \in \mathcal{Z}_f}a\cdot\mu(\underbrace{\{f = a\}}_{f^{-1}(\{a\})})
                    $$
                    \item Ce $f$ ni $\F$-enostvna in $f \geq 0$ definirana, potem je
                    $$
                        \int fd\mu := sup\{\int qd\mu: \text{q je $\F$-enostvna in $q  \leq f$}\}
                    $$
                    \item Ce f ni $\geq 0$, potem je 
                    $$
                        \int fd\mu := \int f^+d\mu - \int f^-d\mu
                    $$
                \end{itemize}
                $\int fd\mu$ pravimo integral $f$ proti $\mu$ (tudi pricakovana vrednost, ce je $\mu$ verjetnostna mera).
                Druge notacije za $\int fd\mu$ so $\mu[f]:=\mu^x[f(x)]:=\int f(x)\mu(dx).$
            \end{definicija}

            Za $A \in \F$ pisemo $\mu[f;A]:= \mu^x[f(x);x \in A]:= \int_A f(x)\mu(dx):= \int f\mathds{1}_Ad\mu$.

            \begin{definicija}
                Integral $f$ proti $\mu$ je dobro definiran $\iff$ $\int f^+d\mu \vee \int f^-d\mu < \infty$.
                $f$ je $\mu$-integrabilen $\iff $ $\int f^+d\mu \vee \int f^-d\mu < \infty$.
            \end{definicija}

            \begin{definicija}
                Naj bo $(\Omega, \F, \mu)$ prostor z mero. 
                $$
                    \mathcal{L}^1(\mu) := \{f \in \F\backslash\mathcal{B}_{\R}: \text{$f$ je $\mu$ integrabilna}\}. 
                $$
                Za $g:\Omega \rightarrow \mathbb{C}$ za katero je $\{\Re g, \Im g\} \subset \mathcal{L}^1(\mu)$ je 
                $\int g d\mu:= \int \Re gd\mu + i\int \Im gd\mu$.
            \end{definicija}

            \begin{izrek}
                Naj bo $(\Omega, \F, \mu)$ prostor z mero. Integral ima sledece lastnosti.
                \begin{enumerate}
                    \item Aditivnost: $\int (f + g)d\mu = \int fd\mu + \int gd\mu$ za $\{f, g\} \subset \F\backslash\mathcal{B}_{[-\infty, \infty]}$ take, da je $\int f^{-1}d\mu \vee \int g^{-1}d\mu < \infty$.
                    \item Integral indikatorja: $\int \mathds{1}_Ad\mu = \mu(A)$ za $\forall A \in \F$. ( V posebnem je $\int 0 d\mu = 0$ in torej $\int fd\mu = \int f^{+}d\mu - \int f^{-}d\mu$ za vse $f \in \F\backslash\mathcal{B}_{[-\infty, \infty]}$)
                    \item Integrali, ki so nic, ki so koncni: Za $f \in \F\backslash\mathcal{B}_{[-\infty, \infty]}$ je $\int f d\mu = 0 \iff \mu(f >0)= 0$ ($f$ je skoraj povsot glede na $\mu$), Ce $\int f d\mu < \infty \Rightarrow \mu(f = \infty) = 0$ ($f < \infty$ skoraj povsot glede na $\mu$.)
                    \item Trikotniska neenakost: $\left|\int fd\mu\right| \leq \int |f|d\mu$ za $\forall f \in \F\backslash\mathcal{B}_{[-\infty, \infty]}$ za katere je $\int |f|d\mu < \infty$.
                    \item Integral ne vidi mnozic z mero 0: Ce je $\int fd\mu = \int gd\mu$ za $\{f, g\} \subset \F\backslash\mathcal{B}_{[-\infty, \infty]}$ za katere je $f = g$ skoraj povsot glede na $\mu$ ($\mu(f \neq g) = 0$)
                    \item Monotonost:
                    \item Homogenost:                
                    
                
                \end{enumerate}
            \end{izrek}


            ... se od lucije za prepisat

            \begin{proof}
                (ib) $\Rightarrow$ (ia) in (ii):
                (ia):
                $$
                    \left(inf_{n \geq m} f_n\right)_{m \in \N} je zaporedje v \F\backslash\mathcal{B}_{[-\infty, \infty]},
                $$
                ki je narascajoce. $f_n^- \leq g \Rightarrow \left(inf_{n \geq m} f_n\right)^- \leq g$. Po (ib) sledi da 
                $$
                    \int \lim_{m \rightarrow \infty}f_m d\mu =  \int \lim_{m \rightarrow \infty} inf_{n \in \N \geq m} f_n d\mu
                    = lim_{m \rightarrow \infty} \int inf_{n \in \N \geq m} fn d\mu \leq liminf_{n \rightarrow \infty} \int f_n d\mu.
                $$
                (ii):
                Uporabim (ia). Za $\left(- |\underbrace{fn - lim_{m\rightarrow \infty} f_m}_{\leq 2g}|\right)_{n \in \N}$ in dobimo 
                $$
                    0 = \int liminf_{n \rightarrow \infty} \left(- |\underbrace{fn - lim_{m\rightarrow \infty} f_m}_{\leq 2g}|\right) d\mu
                    \leq ...
                $$
            \end{proof}

            \begin{posledica}
                Naj bo $(\Omega, \F, \mu)$ prostor z mero in $(f_n)_{n \in \N}$ zaporedje v $\F\backslash\mathcal{B}_{[0, \infty]}$.
                Potem je 
                $$
                    \int \sum_{n \in \N_0}f_n d\mu = \sum_{n \in \N_0}\int f_n d\mu.
                $$
            \end{posledica}

            \begin{proof}
                $$
                    \sum_{n \in \N_0} = lim_{N \rightarrow \infty}\sum_{m = 1}^n f_m (narascajoce v n)
                $$
                Zaradi prejsnjega izreka sledi
                $$
                    \int \sum_{n \in \N_0}f_n d\mu = lim_{N \rightarrow \infty}\int \sum_{m = 1}^n f_m d\mu = 
                    lim_{N \rightarrow \infty}\sum_{m = 1}^n \int f_m d\mu = \sum_{n \in \N_0}\int f_n d\mu.
                $$
            \end{proof}
            
            \begin{posledica}
                Naj bo $(\Omega, \F)$ merljiv prostor in $(\mu_n)_{n \in \N}$ zaporedje mer na $(\Omega, \F)$. Potem je 
                $\sum_{n \in \N}\mu_n$ mera na $(\Omega, \F)$ in za $f \in \F\backslash\mathcal{B}_{[-\infty, \infty]}$ je
                $$
                    \int fd\left(\sum_{n \in \N}\mu_n\right) dobro definirana \iff 
                    \left( \sum_{n \in \N}\int f^+d\mu_n\right) \vee \left(\sum_{n \in \N}\int f^-d\mu_n\right) < \infty.
                $$
                V tem primeru je 
                $$
                    \int f d(\sum_{n \in \N}\mu_n) = \sum_{n \in \N}\int fd\mu_n.
                $$
            \end{posledica}

            \begin{proof}
                Za $A \in \F$ je $\sum_{n \in \N}\mu_n(A) = \int \mu_n(A)c_\N(dn)$. Potem je za zaporedje $(A_m)_{m \in \N}$
                paroma disjunktnih mnozic v $\F$:
                $$
                    \left( \sum_{n \in \N}\mu_n \right)\left(\cup_{m \in \N}A_m\right) = \\
                    \int \mu_n \left(\cup_{m \in \N}A_m\right)c_\N(dn) = \int \sum_{m \in \N}\mu_n(A_m)c_\N(dn) =\\
                    = \sum_{m \in \N} \int \mu_n(A_m)c_\N(dn) = \sum_{m \in \N}\left(\sum_{n\in \N}(\mu_n(A_m))\right) = \\
                    = \sum_{m \in \N}\left[\left(\sum_{n\in\N}\mu_n\right)(A_m)\right]
                $$
            \end{proof}

            Naprej, razred funkcij $f \in \F\backslash\mathcal{B}_{[-\infty, \infty]}$ za katere velja prejsnja posledica je
            $$
                \int f d\left(\sum_{n\in\N}\mu_n\right) = \int \int f d\mu_n c_\N(dm)
            $$
            konveksen stozec zaprt za $\uparrow$ limite, ki svebuje $\mathds{1}_A$ za $A \in \F$.
            Po izreku o monotonem razredu prejsnja posledica velja za vse $f \in \F\backslash\mathcal{B}_{[-0, \infty]}$.

            \begin{definicija}
                Naj bo $(\Omega, \F, \mu)$ prostor z mero in $(\Omega', \F')$ merljiv prostor. Za $ \in \F\backslash\F'$ vpeljemo
                $$
                    f * \mu: \F' \rightarrow [0, \infty], \\
                    (f*\mu)(A') := \mu(f^{-1}(A')) \text{za $A'\in \F'$}.
                $$
                $f*\mu$ oznacimo tudi z $\mu \circ f^{-1}$ ali $\mu_{f}$ in ji recemo potisk $\mu$ po $f$ glede na $\F'$, ali 
                zakon $f$ pod $\mu$ glede na $\F'$ (v primeru, da je verjetnostna).
            \end{definicija}

            \begin{definicija}
                Z $f*\mu$ oznacimo sliko mere $\mu$ pod $f$ glede na $\F$.
            \end{definicija}

            \begin{posledica}
                Naj bo $(\Omega, \F, \mu)$ prostor z mero in $(\Omega', \F')$ merljiv prostor. Za $f \in \F\backslash\F'$ je
                $f*\mu$ mera na $(\Omega', \F')$, ki je koncna ali verjetnostna - kakor je $\mu$. Poleg tega je za $g \in \F'\backslash\mathcal{B}_{[-\infty, \infty]}$
                $$
                    \int g d(f*\mu) \text{dobro definiran} \iff \int g \circ f d\mu text{dobro definiran}.
                $$
                in tedaj je 
                $$
                    \int g d(f*\mu) = \int g \circ f d\mu.
                $$
            \end{posledica}

            \begin{proof}
                $(f*\mu)(\emptyset) = \mu(f^{-1}(\emptyset)) = \mu(\emptyset) = 0.$
                Za zaporedje $(A_n)_{n \in \N}$ paroma disjunktnih mnozic v $\F'$ je 
                $(f*\mu)(\cup_{n\in\N}A_n) = \mu(f^{-1}(\cup_{n \in \N}A_n)) = \mu(\cup_{n\in\N}f^{-1}(A_n)) = \sum_{n\in\N}\mu(f^{-1}(A_n)) = \sum_{n\in\N}(f*\mu)(A_n).$	
                Torej velja da je $f*\mu$ mera na $(\Omega', \F')$.
                Naprej, $(f*\mu)(\Omega') = \mu(f^{-1}(\Omega')) = \mu(\Omega),$ torej je res $f*\mu$ koncna oz. verjetnostna
                $\iff$ je $\mu$ koncna oz. verjetnostna.
            \end{proof}

            Postavimo sedaj
            $$
            \mathcal{M} := \{g \in \F'\backslash\mathcal{B}_{[0, \infty]} \mid \int gd(f* \mu) = \int g \circ f d\mu\}.
            $$
            $\mathds{1}_A \in \mathcal{M} \forall A \in \F',$ kajti $\int \mathds{1}_Ad(f*\mu) = (f*\mu)(A) = \mu(f^{-1}(A)) = \int\mathds{1}_{f^{-1}(A)}d\mu$.
            $\mathcal{M}$ je konveksen stozec.
            dokaz\dots

            $\mathcal{M}$ je zaprta za $\uparrow$ limite.
            dokaz\dots

            Za splosen $g \in \F'\backslash\mathcal{B}_{[-\infty, \infty]}$ je $\{g^+, g^-\} \subset \F'\backslash\mathcal{B}_{[-\infty, \infty]}$,
            zato $\int g^+d(f* \mu) = \int g^+ \circ $ ... to je del dokaza.

            \begin{posledica}
                Naj bo $(X, \Sigma, \mu)$ prostor z mero in $O$ neka odprta podmnozica $\R$ in $f:X \times O \rightarrow \R$. Denimo, da je 
                \begin{itemize}
                    \item $F(., t)$ $\mu$-integrabilna za $\forall t \in O$
                    \item $F(x, .)$ odvedljvia za $\forall x \in X$ 
                \end{itemize}
                Predpostavimo, da $\exists$ $g \in \Sigma\backslash\mathcal{B}_{[o, \infty]}, \int g d\mu < \infty,$ taka da je 
                $$
                    \left|\frac{\partial F}{\partial t}(x, t) \right| \leq g(x) \forall x \in X \forall t \in O.
                $$
                Potem velja 
                \begin{itemize}
                    \item $(X \ni x \rightarrow \frac{\partial F}{\partial t}(x, t))$ je $\mu$-integrabilna
                    \item $(O \ni t \rightarrow \int F(x,t)\mu(dx))$ je odvedljiva in 
                    \item $\frac{d}{dt}\int F(x, t)\mu(dx) = \int \frac{\partial F}{\partial t}(x, t)\mu(dx). \forall t \in O$
                \end{itemize}
            \end{posledica}

            \begin{proof}
                Od tanje prepisi.
            \end{proof}

        \subsubsection{Zamenjava vrstnega reda integracija in povezane teme}
            
            \begin{definicija}
                Naj bosta $(\Omega', \F)$ in $(\Omega', \F')$ merljiva prostora. 
                $$
                    \F \otimes \F' := \sigma_{\Omega \times \Omega'}(\{A \times A' \mid (A, A') in \F \times \F'\})
                $$
                je produktna $\sigma$-algebra $\F$ in $\F'$ $(\mathcal{B}_{\R^n} := \mathcal{B}_{\R} \otimes \cdots \otimes \mathcal{B}_{\R})$ je Borelova
                $\sigma$-algebra na $\R^n$.
            \end{definicija}

            \begin{trditev}
                Za $A \subset \R^2$ in $f:A \rightarrow [-\infty, \infty]$, ki je zvezna je $f \in \mathcal{B}\backslash\mathcal{B}_{[-\infty, \infty]}$.
            \end{trditev}

            \begin{proof}
                Brez dokaza.
            \end{proof}

            \begin{trditev}
                Naj bosta $(\Omega, \F)$ in $(\Omega', \F')$ merljiva prostora. Potem velja:
                \begin{enumerate}
                    \item $\F \otimes \F' $ je najmanjsa $\sigma$-algebra $\mathcal{G}$ na $\Omega \times \Omega$ za katero je $pr_\Omega := (\Omega \times \Omega' \ni (\omega, \omega') \rightarrow w) \in \mathcal{G}\backslash\F$ in $pr_{\Omega'} := (\Omega \times \Omega' \ni (\omega, \omega') \rightarrow \omega') \in \mathcal{G}\backslash\F'$.
                    \item Za $f \in (\F \otimes \F')\backslash\mathcal{B}_{[-\infty, \infty]} \Rightarrow$ $f(\omega, .) \in \F'\backslash\mathcal{B}_{[-\infty, \infty]} \forall \omega \in \Omega$ in $f(., \omega') \in \F\backslash\mathcal{B}_{[-\infty, \infty]} \forall \omega' \in \Omega'$.
                    \item Za vsako $\sigma$-algebro $\mathcal{G}$ na $G$ in $f: G \rightarrow \Omega$ ter $f':G \rightarrow \Omega'$ je $(f, f') \in \mathcal{G}\backslash\F\otimes\F' \iff f \in \mathcal{G}\backslash\F$ in $f'\in \mathcal{G}\backslash\F'.$
                \end{enumerate}
            \end{trditev}

            \begin{proof}
                Brez dokaza.
            \end{proof}

            \begin{izrek}
                Naj bosta $(\Omega, \F, \mu)$ in $(\Omega', \F', \mu')$ prostora z mero ter $\mu$ in $\mu'$ $\sigma$-koncni. Potem velja:
                \begin{enumerate}
                    \item Obstaja natanko ena mera $\ni$ na $\F \otimes \F'$, ki jo oznacimo $\mu \times \mu'$, da je $\ni(A \times A') = \mu(A)\mu'(A') \forall (A, A') \in \F \times \F'$.
                    \item Naj bo $f \in (\F \otimes \F')\backslash\mathcal{B}_{[-\infty, \infty]}$. in naj velja:
                        \begin{enumerate}
                            \item $ f \geq 0$
                            \item $\int|f|d(\mu \times \mu') < \infty$
                            \item $\int\int f^-(\omega, \omega')\mu(d\omega)\mu'(d\omega') and \int\int f^-(\omega, \omega')\mu'(d\omega')\mu(d\omega) < \infty$.
                        \end{enumerate}
                        Potem je $(\Omega \ni \omega' \rightarrow \int f(\omega, \omega')\mu(d\omega)) \in \F'\backslash\mathcal{B}_{[-\infty, \infty]}$ in 
                        $(\Omega' \ni \omega \rightarrow \int f(\omega, \omega')\mu'(d\omega')) \in \F\backslash\mathcal{B}_{[-\infty, \infty]}$. in 
                        $ \int fd(\mu\times\mu') = \int\int f(\omega, \omega')\mu(d\omega)\mu'(d\omega') = \int\int f(\omega, \omega')\mu'(d\omega')\mu(d\omega).$
                \end{enumerate}
            \end{izrek}

            \begin{proof}
                Brez dokaza.
            \end{proof}

            \begin{definicija}
                Meri $\mu \times \mu'$ recemo produktna mera. $(\mathcal{L}^n:= \mathcal{L}\times \cdots \times\mathcal{L}$ 
                pravimo $n$-razsezna Lebesquova mera). 
            \end{definicija}




            
\end{document}