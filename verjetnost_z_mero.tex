\documentclass[a4paper,12pt]{article}
\usepackage[margin=3cm]{geometry}
\usepackage[slovene]{babel}
\usepackage[utf8]{inputenc}
\usepackage[T1]{fontenc}
\usepackage{lmodern}
\usepackage{url}
\usepackage{graphicx}
\usepackage{amsmath}
\usepackage{amssymb}
\usepackage{amsfonts}
\usepackage{hyperref}
\usepackage{amsthm}
\usepackage{pgfpages}
\usepackage{colortbl}
\usepackage{tikz}
\usepackage{array}
\usepackage{amsmath,amsthm, amsfonts,amssymb}
\usepackage{mathtools}
\usepackage{tikz}
\usepackage{dsfont}
\usepackage{mathtools}


% ukazi za matematicna okolja
\theoremstyle{definition} % tekst napisan pokoncno
\newtheorem{definicija}{Definicija}[section]
\newtheorem{primer}[definicija]{Primer}
\newtheorem{zgled}[definicija]{Zgled}
\newtheorem{opomba}[definicija]{Opomba}

\renewcommand\endprimer{\hfill$\diamondsuit$}


\theoremstyle{plain} % tekst napisan posevno
\newtheorem{lema}[definicija]{Lema}
\newtheorem{izrek}[definicija]{Izrek}
\newtheorem{trditev}[definicija]{Trditev}
\newtheorem{posledica}[definicija]{Posledica}

\newcommand{\R}{\mathbb{R}}
\newcommand{\N}{\mathbb{N}}
\newcommand{\E}{\mathbb{E}}
\newcommand{\F}{\mathcal{F}}
\newcommand{\A}{\mathcal{A}}
\newcommand{\B}{\mathcal{B}}
\newcommand{\G}{\mathcal{G}}
\newcommand{\Le}{\mathcal{L}}
\newcommand{\sh}{š}
\newcommand{\ch}{č}
\newcommand{\zh}{ž}
\newcommand{\SH}{Š}
\newcommand{\CH}{Č}
\newcommand{\ZH}{Ž}




\begin{document}

\begin{titlepage}
    UNIVERZA V LJUBLJANI
  
    FAKULTETA ZA MATEMATIKO IN FIZIKO
  
    \vspace{0.5cm}
    Finančna matematika - 1. stopnja
  
    \begin{center}
        \vspace{7cm}
            Anej Rozman
  
        \vspace{0.4cm}
        \textbf{\Large{Verjetnost z mero}}
        \vspace{0.3cm}
  
        Zapiski po predavanjih doc.\@ dr.\@ Matije Vidmarja v študijskem letu 2023/2024.
        (Priporocam da med branjem sledite tudi originalni skripti predavatelja)
    \end{center}
    \vfill
        Ljubljana, 2023     
    \thispagestyle{empty}
\end{titlepage}

\newpage
  
\tableofcontents
   
\newpage
    
\section{Mera}
    \subsection{Merljivost in mere}
        \subsubsection{Merjlive mnozice}
            \begin{definicija}
                Naj bo $\mathcal{A} \subset 2^\Omega$, torej $\mathcal{A} \in 2^{2^\Omega}$ Pravimo, da je $\mathcal{A}$ zaprta za:
                \begin{enumerate}
                    \item $c^\Omega$ (zaprta za komplemente v $\Omega$) $\iff \forall A \in \mathcal{A} \Rightarrow \Omega \backslash A \in \mathcal{A} $
                    \item $\cap $ (zaprta za preseke) $\iff A \cap A' \in \mathcal{A}$ kadarkoli $\{A, A'\} \subset \mathcal{A}$
                    \item $\cup$ (zaprta za unije) $\iff  A \cup A' \in \mathcal{A}$ kadarkoli $\{A, A'\} \subset \mathcal{A}$
                    \item $\backslash$ (zaprta za razlike) $\iff A \backslash A' \in \mathcal{A}$ kadarkoli $\{A, A'\} \subset \mathcal{A}$
                    \item $\sigma\cap$ (zaprta za stevne preseke) $\iff \cap_{n \in \N}A_n \in \mathcal{A}$ kadarkoli je $\left(A_n \right)_{n \in \N}$ zaporedje v $\mathcal{A}$
                    \item $\sigma\cup$ (zaprta za stevne unije) $\iff \cup_{n \in \N}A_n \in \mathcal{A} $ kadarkoli je $\left(A_n \right)_{n \in \N}$ zaporedje v $\mathcal{A}$
                
                \end{enumerate} 
            \end{definicija}
            
            \begin{definicija}($\sigma$-algebra, pod-$\sigma$-algebra in algebra)
                \begin{enumerate}
                    \item$\mathcal{A}$ je $\sigma$-algebra na $\Omega \iff (\Omega, \A)$ je merljiv prostor $\iff \emptyset \in \mathcal{A}$ in $\mathcal{A}$ je zaprta za $c^\Omega$ in $\sigma\cup$. Ce je $\mathcal{A} \ \sigma$-algebra na $\Omega$ potem: $A$ je $\mathcal{A}$-merljiva $\iff A \in \mathcal{A}.$
                    \item $\mathcal{B}$ je pod-$\sigma$-algebra $\mathcal{A} \iff \mathcal{B} \subset \mathcal{A}$ in $\mathcal{B}$ je $\sigma$-algebra na $\Omega$.
                    \item $\mathcal{A}$ je algebra na $\Omega \iff \emptyset \in \mathcal{A}$ in $\mathcal{A}$ je zaprta za $c^\Omega$ in $\cup$.
                \end{enumerate}
            \end{definicija}

        \begin{opomba}
            V primeru ko nimamo podane mnozice $\Omega$ lahko vzamemo $\Omega = \cup\A$ in velja $\A \subset 2^{\Omega}.$
        \end{opomba}

        \begin{zgled}
            $2^\Omega$ je $\sigma$-algebra na $\Omega$ in $\{\emptyset, \Omega\}$ je $\sigma$-algebra na $\Omega$. Klicemo ju diskretna in trivialna $\sigma$-algebra.
        \end{zgled}

        \begin{zgled}
        Naj bo $A \subset \Omega$. Potem je $ \sigma_\Omega A:=\{\emptyset, A, A^C, \Omega\}$ $\sigma$-algebra na $\Omega$.
        \end{zgled}

        \begin{zgled}
        $\sigma_\Omega^{ccc} := \{ A \in 2^\Omega: A \ \text{je stevna ali} \ \Omega \backslash A \ \text{je stevna}\}$ je $\sigma$-algebra na $\Omega$.
        To je oznaka za stevno kostevno $\sigma$-algebro na $\Omega$. Seveda je $\sigma_\Omega^{ccc}$ = $2^\Omega$ razen ce $\Omega$ ni stevna.
        \end{zgled}

        \begin{zgled}
        Naj bo $\mathcal{P}$ particija $\Omega$ (torej $\mathcal{P} \subset 2^\Omega$ in $\mathcal{P}$ je druzina paroma disjunktnih mnozic, ki pokrije $\Omega$).
        Potem je $\sigma \mathcal{P} := \{\cup R \mid R \subset P \ in \ (R \ ali \ P\backslash R \ je \ stevna)\}$ je sigma algebra na $\Omega$.
        \end{zgled}

        \begin{trditev}
            Naj bo $\A \subset 2^\Omega$ zaprta za $c^\Omega$ in naj bo $\emptyset \in \A$. Potem je $\A$ $\sigma$-algebra na $\Omega$ ce in samo ce je $\A$ zaprta za $\sigma\cup$, in v tem primeru je $\A$ zaprta za $\cap$ in $\cup$ in $\backslash$.
        \end{trditev}
        \begin{proof}
            Sledi iz de Morganovih zakonov:
            $$
            \cap_{n \in \N} A_n = \Omega \backslash \left(\cup_{n \in \N}(\Omega\backslash A_n) \right)
            $$
            $$
            \cup_{n \in \N} A_n = \Omega \backslash \left(\cap_{n \in \N}(\Omega\backslash A_n) \right)
            $$
            Zaprtost $\sigma$-algebre $A$ na $\Omega$ za:
            \begin{enumerate}
                \item $\cap$: $A \cap B = A \cap B \cap \Omega \cap \Omega \cdots \in \A$
                \item $\cup$: $A \cup B = A \cup B \cup \emptyset \cup \emptyset \cdots \in \A$
                \item $\backslash$: $A \backslash B = A  \cap \left( \Omega \backslash B \right) \in \A$
            \end{enumerate}

        \end{proof}

        \subsubsection{Mere}
            \begin{definicija}(Mera)
                Naj bo $\left( \Omega, \F \right)$ merljiv prostor in $\mu: \F \rightarrow [0, \infty] $. \\
                $\mu$ je mera na $\left( \Omega, \F \right)$ natanko tedaj ko: 
                \begin{enumerate}
                    \item $\mu(\emptyset) = 0$
                    \item $\mu$ je stevno aditivna: za $\forall$ zaporedje $(A_n)_{n \in \N} \subset \F$ paroma disjunktnih mnozic je $\mu(\cup_{n \in \N} A_n) = \sum_{n \in \N} \mu(A_n)$
                \end{enumerate}
            \end{definicija}

            Za mero $\mu$ na $\left( \Omega, F \right)$ recemo da je: 
            \begin{enumerate}
                \item koncna $\iff \mu(\Omega) < \infty$
                \item verjetnostna mera $\iff \mu(\Omega) = 1$
                \item $\sigma$-koncna $\iff \exists (A_n)_{n \in \N} \subset \F: \Omega = \cup_{n \in \N} A_n \ \text{in} \ \mu(A_n) < \infty \ \forall n \in \N$
            \end{enumerate}

            \begin{definicija}(Prostor z mero)
                $\left( \Omega, F, \mu \right)$ je prostor z mero $\iff \mu$ je mera na $(\Omega, F)$
            \end{definicija}

            \begin{definicija}
                Ce je $\left( \Omega, F, \mu \right)$ merljiv prostor. Potem za $A \in \F$ recemo:
                \begin{enumerate}
                    \item $A$ je $\mu$-zanemarljiva $\iff \mu(A) = 0$
                    \item $A$ je $\mu$-trivialna $\iff \mu(A) = 0$ ali $\mu(\Omega\backslash A) = 0$
                \end{enumerate}
            \end{definicija}

            Ce imamo neko lastnost $P(\omega)$ v $\omega \in A$, potem: 
            \begin{enumerate}
                \item $P(\omega)$ drzi $\mu$-skoraj povsod v $\omega \in A \iff A_{\neg P} := \{\omega \in A: \neg P(\omega)\}$ je $\mu$-zanemarljiva.       
                \item $P(\omega)$ drzi $\mu$-skoraj gotovo v $\omega \in A \iff A_{\neg P} := \{\omega \in A: \neg P(\omega)\}$ je $\mu$-zanemarljiva in $\mu$ je verjetnost.
                \item $P$ drzi $\mu$-s.p. na $A \iff P(\omega)$ drzi $\mu$-s.p. v $\omega \in A$. 
                \item $P$ drzi $\mu$-s.g. na $A \iff P(\omega)$ drzi $\mu$-s.g. v $\omega \in A$. 
            \end{enumerate}
            \begin{zgled}
                Nicelna mera na $\F$ (torej preslikava $\mu(A) \rightarrow 0, \ \forall A \in \F)$ je vedno mera na katerikoli $\sigma$-algebri.
            \end{zgled}      
            
            \begin{zgled}
                Ce definiramo $c_\Omega:2^\Omega \rightarrow [0, \infty]$ kot $c_\Omega(A) := |A|$ ce je $A$ koncna podmnozica $\Omega$ in $c_\Omega(A) := \infty$ ce je $A$ neskoncna podmnozica $\Omega$, je $c_\Omega$ tako imenovana stevna mera na $\Omega$. Ko je $\Omega$ koncna in neprazna, potem je $\frac{c_\Omega}{|\Omega|}$ verjentostna mera na $\Omega$.  
            \end{zgled}

            \begin{zgled}
                Ce definiramo $\delta_x:2^\Omega \rightarrow [0, \infty]$ za fiksen $x \in \Omega$, tako da za $A \in 2^\Omega, \delta_x(A):=0$ ce $x \notin A$ in $\delta_x(A):=1$ ce $x \in A$, potem je $\delta_x$ tako imenovana Diracova mera za $x$. Katerakoli podmnozica $\Omega \backslash \{x\}$ je $\delta_x$-zanemarljiva.                
            \end{zgled}
            
            \begin{trditev}(Lastnosti mere)
                Naj bo $\mu$ mera na merljivem prostoru $\left( \Omega, \F \right)$. Potem velja naslednje:
                \begin{enumerate}
                    \item $\mu$ je aditivna: $\mu(A \cup B) = \mu(A) + \mu(B)$ kadarkoli $A \cap B = \emptyset$ in $\{A, B\} \in \F$.
                    \item $\mu$ je monotona: $A \subset B$ in $ \{A, B\} \in \F$ $ \Rightarrow \mu(A) \leq \mu(B)$.
                    \item $\mu$ je zvezna od spodaj: $\mu(\cup_{n \in \N}A_n) = \uparrow\text{-}\lim_{n \rightarrow \infty}\mu(A_n)$ kadarkoli je $(A_n)_{n \in \N}$ narascajoce zaporedje v $\F$.
                    \item $\mu$ je stevno subaditivna: $\mu(\cup_{n \in \N}A_n) \leq \sum_{n \in \N}\mu(A_n)$ kadarkoli je $(A_n)_{n \in \N}$ zaporedje v $\F$.
                    \item Predpostavimo da je $\mu$ koncna. $\mu(\Omega\backslash A) = \mu(\Omega) - \mu(A)$ za vse $A \in \F$. Se vec, $\mu$ je zvezna od zgoraj: $\mu(\cap_{n \in \N}A_n) = \downarrow\text{-}\lim_{n \rightarrow \infty}\mu(A_n)$ kadarkoli je $(A_n)_{n \in \N}$ padajoce zaporedje v $\F$.
                    \item Za $A \in \F$ je $\F|_A:=\{B \cap A: B \in \F\}$; potem je $\mu_A:= \mu_{\F|_A}$ mera na $\F|_A$. Imenuje se restrikcija/skrcitev mere $\mu$ na $A$.  
                \end{enumerate}             
            \end{trditev}

            \begin{proof}
                \begin{enumerate}
                    \item $\left(A, B, \emptyset, \emptyset, \cdots \right)$ je zaporedje medseboj disjunktnih mnozic v $\F$, torej po stevni aditivnosti velja $\mu(A \cup B \cup \emptyset \cup \cdots) = \mu(A) + \mu(B) + \mu(\emptyset) + \cdots = \mu(A) + \mu(B)$.
                    \item $B = A \cup (B \backslash A)$ in uporabimo koncno aditivnost (1.).
                    \item $\left(A_1, A_2\backslash A_1, A_3\backslash A_2, \cdots \right)$ je zaporedje paroma disjunktnih mnozic v $\F$ z unijo $\cup_{n \in \N}A_n$. Uporabimo stevno aditivnost in za tem se koncno aditivnost.
                    \item $\left(A_1, A_2\backslash A_1, A_3\backslash (A_1 \cup A_2), \cdots \right)$ je zaporedje paroma disjunktnih mnozic v $\F$ z unijo $\cup_{n \in \N}A_n$. Uporabimo stevno aditivnost in za tem se monotonost (2.).
                    \item Prvi del sledi iz koncne aditivnosti (1.). Za mnozici vzamemo $A \in \Omega$ in  $B = \Omega \backslash A$. Dobimo $\mu(\Omega) = \mu(A) + \mu(\Omega \backslash A)$. Drugi del sledi iz zveznosti od spodaj (3.) tako da jo uporabimo na $\left(\Omega\backslash A_n \right)_{n \in \N}$. $\mu(\Omega) - \mu(\cap_{n \in \N}A_n) = \mu\left( \cup_{n \in \N}\Omega\backslash A_n\right) = \lim_{n \rightarrow \infty}\mu(\Omega\backslash A_n) = $ $\lim_{n \rightarrow \infty}\mu(\Omega) - \mu(A_n) = \mu(\Omega) - \lim_{n \rightarrow \infty}\mu(A_n)$ $ \Rightarrow \mu(\cap_{n \in \N}A_n) = \lim_{n \rightarrow \infty}\mu(A_n).$
                    \item Preveriti moramo, da je $\F|_A$ $\sigma$-algebra na $A$. Kasneje bomo videli da je $\F|_A = 2^A \cap \F$ in bo dokaz sledil iz tega.
                \end{enumerate}
            \end{proof}

            \begin{zgled}(Borel-Cantellijeva lema)
                Naj bo $\left( \Omega, \F, \mu \right)$ merljiv prostor in $(A_n)_{n \in \N}$ zaporedje v $\F$, da velja $\sum_{n \in \N}\mu(A_n) < \infty$. Potem je $\mu(\limsup_{n \rightarrow \infty}A_n) = 0$.
            \end{zgled}

            \begin{proof}
                Zaporedje mnozic $\{\cup_{n=N}^\infty A_n\}_{N=1}^\infty$ je padajoce:
                $$
                \cup_{n=N}^\infty A_n \supseteq \cup_{n=N+1}^\infty A_n \supseteq \cdots \supseteq \cup_{n=M}^\infty A_n \supseteq \cdots \supseteq \limsup_{n \rightarrow \infty}A_n = \cap_{N=1}^\infty \cup_{n=N}^\infty A_n
                $$
                Po zveznosti od zgoraj (5.) sledi:
                $$
                \mu(\limsup_{n \rightarrow \infty}A_n) = \mu(\cap_{N=1}^\infty \cup_{n=N}^\infty A_n) = \lim_{N \rightarrow \infty}\mu(\cup_{n=N}^\infty A_n) \leq \lim_{N \rightarrow \infty}\sum_{n=N}^\infty \mu(A_n) = 0
                $$
            \end{proof}

            \begin{zgled}
                Ce je $P$ vejretnost na $\left( \Omega, \F \right)$, potem je $P^{-1}(\{0, 1\})$ pod-$\sigma$-algebra na $\F$. Tako imenovana $P$-trivialna $\sigma$-algebra.
            \end{zgled}

        \subsubsection{Merjlive preslikave in generirane $\sigma$-algebre}
            
            \begin{definicija}(Generirana $\sigma$-algebra)
                Naj bo $\A \subset 2^\Omega$; potem $$\sigma_\Omega(\A):=\cap\{\F \in 2^{2^\Omega}: \F \ \text{je} \ \sigma\text{-algebra na} \ \Omega \ \text{in} \ \A \subset \F \}$$ imenujemo $\sigma$-algebra generirana na $\Omega$ z $\A$. Je najmanjsa $\sigma$-algebra, ki vkljucuje druzino podmnozic $\A$.
            \end{definicija}

            \begin{opomba}
                $2^\Omega$ je gotovo $\sigma$-algebra na $\Omega$, ki vsebuje $\A$, torej je $\sigma_\Omega(\A)$ neprazna.
            \end{opomba}

            Za dve $\sigma$-algebri $\mathcal{B}_1$ in $\mathcal{B}_2$ na $\Omega$ je mnozica $\mathcal{B}_1 \vee \mathcal{B}_2 := \sigma_\Omega\left(\mathcal{B}_1 \cup \mathcal{B}_2\right)$ skupek $\mathcal{B}_1$ in $\mathcal{B}_2$. Bolj splosno
            za druzino $\left(\mathcal{B}_\lambda\right)_{\lambda \in \Lambda} \sigma$-algeber na $\Omega$ pravimo, da je $\vee_{\lambda \in \Lambda}\mathcal{B}_\lambda := \sigma_\Omega\left(\cup_{\lambda \in \Lambda}\mathcal{B}_\lambda\right)$ njen skupek.

            \begin{opomba}
                Razlog zakaj so generirane $\sigma$-algebre pomembne v teoriji mere je, ker le redko lahko eksplicitno
                podamo vse elemente $\sigma$-algebre, ki bi jo zeleli, ampak pogosto lahko eksplicitno podamo njene generatorje.  
            \end{opomba}

            \begin{definicija}(Zacetna in koncna struktura)
                Naj bo $f:\Omega \rightarrow \Omega'$. Za podano $\sigma$-algebro $\F'$ na $\Omega'$  definiramo $$\sigma^{\F'}(f):= f^{-1}(\F'):=\{f^{-1}(A'):A' \in \F'\},$$
                zacetno strukturo za $f$ glede na $\F'$. (oziroma $\sigma$-algebra generirana z $f$ glede na $\F'$).

                Za podano $\sigma$-algebro $\F$ na $\Omega$ definiramo $$\sigma^{\Omega'}_{\F}(f):=\{A' \in 2^{\Omega'}: f^{-1}(A') \in \F\},$$
                koncno strukturo za $f$ na $\Omega'$ glede na $\F$.
            \end{definicija}

            \begin{definicija}
                Za dano $\sigma$-algebro $\F'$ na $\Omega'$ in $\sigma$-algebro $\F$ na $\Omega$ pravimo da je $f$ $\F/\F'$-merljiva preslikava $\iff f^{-1}(A')\in \F$ za vse $A' \in \F'$.
            \end{definicija}

            \begin{opomba}
                \begin{enumerate}
                    \item V notacijah $\sigma_\Omega(\A)$ in $\sigma^{\F'}(f)$ spuscamo $\Omega$ in $\F'$ kadar je to jasno iz konteksta. Torej pisemo preprosto $\sigma(\A)$ in $\sigma(f)$.
                    \item V primeru ko je zaloga vrednosti $f$ stevna in nimamo podane ne $\F'$ ali $\Omega'$ za $\Omega'$ vazamemo $Z_f$ in za $\F'$ vazamemo $2^{\Omega'}$.
                    \item Izmed objektov, ki smo ju uvedli v definiciji 1.21 je zacetna struktura bolj sugestivna/pomembna.
                \end{enumerate}
            \end{opomba}

            \begin{definicija}
                Za dano $\sigma$-algebro $\F$ na $\Omega$ in $\F'$ na $\Omega'$ definiramo $$\F/\F':= \{g \in \Omega'^\Omega: g \ \text{je} \ \F/\F'\text{-merljiva}\}.$$
            \end{definicija}

            \begin{zgled}
                Konstantna funkcija je vedno merljiva, ne glede na $\sigma$-algebro. Za poljubno $\sigma$-algebro $\F$ na $\Omega$ je $id_\Omega \in \F/\F'$.
            \end{zgled}

            \begin{definicija}(Indikator)
                Za $A \subset \Omega$ definiramo $\mathds{1}_{A_\Omega}:\Omega\rightarrow\{0, 1\}$:
                $$\mathds{1}_{A_\Omega}(x) := 
                    \begin{cases}
                        1, & \ x \in A \\
                        0, & \text{sicer}
                    \end{cases}$$
                Funkciji pravimo indikator mnozice $A$ v $\Omega$. Pisali bomo $\mathds{1}_A$ in predvidevali da se $\Omega$ da razbrati iz konteksta.
            \end{definicija}

            \begin{zgled}
                Naj bo $A \subset \Omega$. Potem $\sigma^{2^{\{0, 1\}}}(\mathds{1}_A) = \sigma_\Omega (A) = \{\emptyset, \Omega, A, \Omega\backslash A\}$. Ce je nadaljno $\F$ $\sigma$-algebra na $\Omega$ potem je  $\mathds{1}_A \in \F/2^{\{0, 1\}} \iff A \in \F$ 
            \end{zgled}

            \begin{trditev}
                Naj bodo $\F, \mathcal{G}, \mathcal{H}$ $\sigma$-algebre (vsaka na svoji mnozici). Naj bo $f \in \F/\mathcal{G}$ in $g \in \mathcal{G}/\mathcal{H}$. Potem je $g \circ f \in \F/\mathcal{H}.$ Z besedami to pomeni: kompozitumi merljivih preslikav so merljive preslikave.
            \end{trditev}

            \begin{proof}
                $(g \circ f)^{-1}(H) = f^{-1}(g^{-1}(H))$ za $\forall H$ $\in \mathcal{H}$
            \end{proof}

            \begin{trditev} (Lastnosti preslikav)
                Naj bo $f: \Omega \rightarrow \Omega'$.
                \begin{enumerate}
                    \item Naj bo $\F'$ $\sigma$-algebra na $\Omega'$. $\sigma^{\F'}(f)$ je $\sigma$-algebra na $\Omega$; je najmanjsa $\sigma$-algebra $\mathcal{F}$ na $\Omega$ da velja $f \in \mathcal{F}/\F'$.
                    \item Naj bo $\F$ $\sigma$-algebra na $\Omega$. $\sigma_{\F}^{\Omega'}(f)$ je $\sigma$-algebra na $\Omega'$; je najvecja $\sigma$-algebra $\mathcal{F}'$ na $\Omega'$ da velja $f \in \mathcal{F}/\F'$.
                    \item Naj bo $\F'$ $\sigma$-algebra na $\Omega'$ in $\F$ $\sigma$-algebra na $\Omega$, potem $f \in \F/\F' \iff \sigma^{\F'}(f) \subset \F \iff \sigma_{\F}^{\Omega'}(f) \supset \F'.$
                    \item Naj bo $\A' \subset 2^{\Omega'}$. $\sigma_{\Omega'}(\A')$ je najmanjsa $\sigma$-algebra na $\Omega'$, ki ima $\A'$ za svojo podmnozico. Naj bo $\F$ $\sigma$-algebra na $\Omega$, potem je $f \in \F/\sigma_{\Omega'}(\A') \iff (f^{-1}(A') \in \F$ za vse $A' \in \A'$). Natancno zapisemo $\sigma^{\sigma_{\Omega'}(\A')}(f) = \sigma_\Omega(\{f^{-1}(A'):A' \in \A'\})$. V posebnem je $f^{-1}(\sigma_{\Omega '}(\A ')) = \sigma_{\Omega}(f^{-1}(\A').$
                \end{enumerate}
            \end{trditev}

            \begin{proof}
                \begin{enumerate}
                    \item $f^{-1}(\F')$ je $\sigma$-algebra na $\Omega$: $\forall \emptyset : f^{-1}(\emptyset) \in f^{-1}(\F');$ za $A'\in \F'$ je $\Omega \backslash f^{-1}(A') = f^{-1}(\Omega \backslash A') \in f^{-1}{\F'};$ za zaporedje $(A_i)_{i \in \N}$ je $\cup_{i \in \N}f^{-1}(A_i') = f^{-1}(\cup_{i \in \N}A_i') \in f^{-1}(\F').$ Drugi del je jasen
                    \item Podoben dokaz kot 1.
                    \item Je direktna posledica definicije zacetne in koncne strukture.
                    \item Prvi del: je jasen iz defnicije generirane $\sigma$-alebre. \\
                    Drugi del: $(\Rightarrow):$ je jasen iz definicije generirane $\sigma$-algebre. \\
                    $(\Leftarrow):$ $A' \subset \sigma_{\F}^{\Omega '}(f)$ torej sledi $\sigma_{\Omega'}(A')\subset \sigma_{\F}^{\Omega '}(f)$ in zato sledi $f \in \F\backslash\sigma_{\Omega'}(A'). $ \\
                    Tretji del: V drugem delu vzamemo $\F = \sigma_{\Omega}(f^{-1}(A'))$, kar nam da $f \in \sigma_{\Omega}(f^{-1}(A'))\backslash\sigma_{\Omega'}(A')$. Opazimo, da je $f^{-1}(A')\subset f^{-1}(\sigma_{\Omega'}(A'))$ in zato po tocki 1. velja $\sigma_{\Omega}(f^{-1}(A')) \subset f^{-1}(\sigma_{\Omega'}(A'))$.
                \end{enumerate}
            \end{proof}

            \begin{opomba}
                Tocka 4 nam pove da je dovolj da dokazemo  merljivo lastnost na mnozici generatorjev. Se en nacin zapisa $f \in \F/\sigma_{\Omega'}(\A') \iff (f^{-1}(A') \in \F$ za vse $A' \in \A'$). Natancno zapisemo $\sigma^{\sigma_{\Omega'}(\A')}(f) = \sigma_\Omega(\{f^{-1}(A'):A' \in \A'\})$ je $f^{-1}(\sigma_{\Omega'}(\A')) = \sigma_\Omega(f^{-1}(\A'))$, kar bomo interpretirali kot operacija dobivanja praslik in generiranih $\sigma$-algeber komutirati.
            \end{opomba}

            \begin{definicija}
                Pisemo $\A|_A := \{A'\cap A : A'\in \A\}$ za sled $\A$ na $A$.
            \end{definicija}

            \begin{opomba}
                Ce je $\F$ zaprta za $\cap$ in $A \in \F$, potem je $\F|_A = \F \cap 2^A$.
            \end{opomba}

            \begin{posledica}
                Naj bo $\A \subset 2^\Omega$. Ce je $A \subset \Omega$, potem $$ \sigma_\Omega(\A)|_A = \sigma_A(\A|_A);$$ v primeru ce je $\A$ $\sigma$-algebra na $\Omega$, potem je $\A|_A$ $\sigma$-algebra na $A$.
            \end{posledica}

            \begin{proof}
                Po prejsnji trditvi (1. tocka) je $\sigma_\Omega(\A)|_A = \sigma^{\sigma_\Omega(\A)}(id_A)$ $\sigma$-algebra na $A$ ki vsebuje $\A|_A$, torej velja $\sigma_A(\A|_A) \subset \sigma_\Omega(\A|_A).$ Po 2. tocki je $\mathcal{C} := \{ C \in 2^\Omega: C \cap A \in \sigma_A(\A|_A)\} = \sigma_{\sigma_A(\A|_A)}^\Omega(id_A)$ $\sigma$-algebra na $\Omega$, ki vsebuje $\A$, torej $\sigma_\Omega(\A) \subset \mathcal{C}$, torej $\sigma_\Omega(\A)|_A \subset \sigma_A(\A|_A).$

            \end{proof}


            \begin{opomba}
                Kako lahko v sposnem dolocimo $\sigma_\Omega(\A)$? Zacnemo z $\A$ karkoli kar mora biti v $\sigma_\Omega(\A)$ da zadosca pogojem $\sigma$-algebre, vse komplemente, stevne unije, $\emptyset$, $\Omega$ in stevne unije teh itd. Po tem postopku dokazemo, da je to kar imamo $\sigma$-algebra.
            \end{opomba}

            \begin{zgled}
                Naj bosta $\{E, F\} \subset 2^\Omega.$ Potem mora $\sigma_\Omega(\{E, F\})$ vsebovati $\{\emptyset, E, F, E\backslash F, E \cap F, \cdots\}$ (Particije na $\Omega$ inducirane z $E, F$). Torej $\sigma_\Omega(\{E, F\}) \supset \sigma_{\Omega}(\mathcal{P})$. Ampak $\sigma_{\Omega}(\mathcal{P})$ je $\sigma$-algebra na $\Omega$, ki vsebuje $\{E, F\}$, torej $\sigma_\Omega(\{E, F\}) = \sigma_{\Omega}(\mathcal{P}) = \sigma\mathcal{P}$ iz zgleda 1.8.
            \end{zgled}

            \begin{trditev}
                Naj bo $f:\Omega \rightarrow \Omega'$ in naj bo $\F$ $\sigma$-algebra na $\Omega$ ter $\F'$ $\sigma$-algebra na $\Omega'$.
                \begin{enumerate}
                    \item Ce je $A' \subset \Omega$ taksna, da $f:\Omega \rightarrow A'$, potem je $f \in \F/\F'$ natanko tedaj ko je $f \in \F/(\F'|_{A'})$.
                    \item Za $A \subset \Omega$, ce je $f \in \F/\F'$, je $f|_{A} \in (\F|_A)/\F'$.
                    \item Ce za zaporedje $(A_i)_{i \in \N}$ v $\F$, za katerega je $\Omega = \cup_{i \in \N}A_i$, velja $f|_{A_{i}} \in (\F|_{A_{i}})/\F'$ za vsak $i \in \N$, potem $f \in \F/\F'$.
                \end{enumerate}
            \end{trditev}

            \begin{proof}
                \begin{enumerate}
                    \item Za $H'\in \F'$ je $f^{-1}(H') = f^{-1}(H'\cap A').$
                    \item Za $F' \in \F'$ je $(f|_A)^{-1}(F') = A \cap f^{-1}(F') \in \F|_A$.
                    \item Za $F'\in \F'$ je $f^{-1}(F') = \cup_{i \in \N}f^{-1}(F')\cap A_i \underbrace{=}_{\cup_{i \in \N}A_i = \Omega} \cup_{i \in \N}\underbrace{(f|_{A_i})^{-1}(F')}_{\in \F|_{A_i} = \F\cap 2^{A_i} \subset \F}$
                \end{enumerate}
            \end{proof}

            \begin{opomba}
                Tocki 1. in 2. pomenita da se merljivost obnasa lepo pod omejitvami. Tocka 3. pa nam pove da je lahko merljivost preverjena "lokalno". 
            \end{opomba}

        \subsubsection{Borelove mnozice na razsirjeni realni osi in Borelova merljivost numericnih funkcij}   
            \begin{definicija}
                Naj bo $[-\infty, \infty] := \R \cup \{-\infty, \infty\}$ razsirjena realna os, opremljena z naravno relacijo $\leq$. Vpeljemo druzino mnozic $\mathcal{B}_{[-\infty, \infty]}:= \sigma_{[-\infty, \infty]}(\{[-\infty, a]:a \in \R\})$, ki ji pravimo Borelova $\sigma$-algebra na $[-\infty, \infty]$. Za $A \subset [-\infty, \infty]$ vpeljemo druzino mnozic $\mathcal{B}_A = \mathcal{B}_{[-\infty, \infty]}|_A$, ki ji pravimo Borelova $\sigma$-algebra na $A$. 
            \end{definicija}

            \begin{opomba}
                Funkcije, ki so merljive glede na $\mathcal{B}_{[-\infty, \infty]}$ na kodomeni, so nekako natanko tiste, ki se "lepo" 
                obnasajo s stalisca integracije. Pričakovanje tega, kar sledi, nam je všeč tudi zato, ker na $(\R, \mathcal{B}_\R)$ 
                lahko definiramo prijetno - netrivialno translacijsko invariantno - tako imenovano Lebesquovo mero. Prav tako lahko 
                trdimo, da je reči, da je numerična preslikava $f$ merljiva (z vidika merjenja zanimiva),
                 treba vsaj izmeriti množice $\{f \leq a\}:= f^{-1}{[-\infty, a]}$ za vsak $a \in \R$ (zlasti, 
                ob malce predvidevanju vsebine drugega dela teh zapisov, za naključno spremenljivko $X$ bi 
                želeli biti sposobni povedati, kaj je verjetnost dogodkov $\{X \leq a\}$ za $a \in \R$).
            \end{opomba}

            \begin{zgled}
                Vsi intervali in stevne podmnozice $[-\infty, \infty]$ pripadajo $\mathcal{B}_{[-\infty, \infty]}$. Prav tako vse zaprte in odprte podmnozice $[-\infty, \infty]$ pripadajo $\mathcal{B}_{[-\infty, \infty]}$. Ce je $A \subset [-\infty, \infty]$ stevna, potem je $\mathcal{B}_A = 2^A.$
            \end{zgled}
        
            \begin{definicija}
                Ce $f$ slika v $[-\infty, \infty]$ (je numericna), potem: 
                \begin{enumerate}
                    \item $\sigma(f) = \sigma^{\mathcal{B}_{[-\infty, \infty]}}(f)$.
                    \item Za $\sigma$-algebro $\F$ na $D_f$ recemo da $f$ je  $\F$-merljiva $\iff f \in \F/\mathcal{B}_{[-\infty, \infty]}$.
                    \item Za $g:D_f \rightarrow [-\infty, \infty]; g \land f :=$ min$\{g, f\}$,  $g \lor f:=$ max$\{g, f\}$, $f^+:=$ max$\{f, 0\}$ in  $f^-:=$ max$\{-f, 0\}$.
                \end{enumerate}
            \end{definicija}

            \begin{zgled}
                    $\mathcal{B}_\R = \sigma_\R(\{(-\infty, a]: a \in \R\})$. Posledicno po trditvah 1.29 in 1.36 za $\sigma$-algebro $\F$ na $\Omega$ in $f:\Omega \rightarrow \R$, $f$ Borelovo merljiva $\iff$ $f \in \F/\mathcal{B}_\R \iff \{f \leq a\}:= f^{-1}((-\infty, a]) \in \F$ za $\forall a \in \R.$
            \end{zgled}
            \pagebreak
            \begin{definicija} (Aritmetika z $\infty$)
                 \begin{enumerate}
                    \item $0\cdot (\pm\infty) := 0$
                    \item $\infty + (-\infty) := 0$
                 \end{enumerate}
                 Preostanek aritmetike v $[-\infty, \infty]$ vpeljemo naravno, npr. $a \cdot \infty = sgn(a)\infty$ za $a \in [-\infty, \infty]\backslash\{0\}$, $a + \infty = \infty$ itd.
            \end{definicija}

            \begin{trditev}
                Za $A \subset [-\infty, \infty]$ in $f:A \rightarrow [-\infty, \infty]$ zvezna, potem $f \in \mathcal{B}_A\backslash\mathcal{B}_{[-\infty, \infty]}.$ Ce je $\F$ $\sigma$-algebra 
                in $\{f, g\} \subset \F\backslash\mathcal{B}_{[-\infty, \infty]},$ potem je $\{f+g, fg\} \subset \F\backslash\mathcal{B}_{[-\infty, \infty]}$ in $\{\{f=g\}, \{f\subset g\}, \{f \leq g\}\} \subset \F$.
            \end{trditev}

            \begin{proof}
                Brez dokaza.
            \end{proof}

            \begin{trditev}
                Naj bo $\F$ $\sigma$-algebra in $\left(f_n\right)_{n \in \N}$ zaporedje v $\F\backslash\mathcal{B}_{[-\infty, \infty]}.$
                Potem je $$\{\sup_{n \in \N}f_n, \inf_{n \in \N}f_n, \limsup_{n \rightarrow \infty}f_n, \liminf_{n \rightarrow \infty}f_n \} \subset \F\backslash\mathcal{B}_{[-\infty, \infty]}.$$
                Ce je $f_n \geq 0$ za $n \in \N$, je $\sum_{n \in \N}f_n \in \F\backslash\mathcal{B}_{[0, \infty]}.$
            \end{trditev}

            \begin{proof}
                $\mathcal{B}_{[-\infty, \infty]} = \sigma_{[-\infty, \infty]}(\{[-\infty, a]: a\in\R\})$. Za $a \in \R$ 
                $\left(\sup_{n \in \N}f_n\right)^{-1}\left([-\infty, a]\right) = \{\sup_{n \in \N}f_n \leq a\} = \{f_n \leq a \ \forall n \in \N\} =
                 \cap_{n \in \N}{f_n \leq a}= \cap_{n \in \N}\underbrace{f_n^{-1}([-\infty, a])}_{\in \F \because f_n \in \F\backslash\mathcal{B}_{[-\infty, \infty]}}$
                 Da je $\inf_{n \in \N}f_n \in \F\backslash\mathcal{B}_{[-\infty, \infty]}$ utemeljimo tako, da zapisemo $\inf_{n \in \N} = -\sup_{n \in \N} - f_n$ in opazimo, da je $-id_{[-\infty, \infty]} \in \mathcal{B}_{[-\infty, \infty]}\backslash\mathcal{B}_{[-\infty, \infty]}$
                 $\limsup$ in $\liminf$ sta kombinaciji $\sup$ in $\inf$. Koncno v primeru, da je $f_n \geq 0$ $\forall n \in \N$ je $\sum_{n\in\N}f_n = \lim_{n \rightarrow \infty}\sum_{k = 1}^{n}f_k = \limsup_{n \rightarrow \infty}\underbrace{\sum_{k = 1}^nf_k}_{\in \F\backslash\mathcal{B}_{[-\infty, \infty]} \text{po trditvi 1.44}}$
                 Kar nam da $\sum_{n \in \N}f_n \in \F\backslash\mathcal{B}_{[0, \infty]}$.
            \end{proof}

            \begin{posledica}
                Naj bo $\F$ $\sigma$-algebra, potem je $\{\max(f, g), \min(f, g), f^+, f^-, |f|\} \subset \F\backslash\mathcal{B}_{[-\infty, \infty]}$ za $\{f, g\} \subset \F\backslash\mathcal{B}_{[-\infty, \infty]}.$ Poleg tega je 
                $\{\{f_n \ \text{konvergira, ko gre} \ n \rightarrow \infty\},$ 
                $ \{f_n \text{konvergira k vrednosti iz} \ \R \ \text{ko gre} \ n \rightarrow \infty\}, \{\lim_{n \rightarrow \infty}f_n = f_0\}\} \subset \F$ za vsako
                zaporedje $(f_n)_{n \in \N_0}$ v $\F\backslash\mathcal{B}_{[-\infty, \infty]}.$
            \end{posledica}

            \begin{proof}
                Brez dokaza.
            \end{proof}

        \subsubsection{Argumenti monotonega razreda}
            IDEJA: Zelimo dokazati tridtev, ki se tice vseh funkcij iz $\F\backslash\mathcal{B}_{[-\infty, \infty]}.$ Najprej pokazemo trditev za 
            $\mathds{1}_A, A \in \F$. $\rightarrow$ izrek o monotonem razredu $\rightarrow$ trditev velja v splosnem.

            \begin{definicija}
                Naj bo $\F$ $\sigma$-algebra na $\Omega$ in $f:\Omega\rightarrow [0, \infty]$.
                f je $\F$-enostavna $\iff$ $f \in \F\backslash\mathcal{B}_{[-\infty, \infty]}$ in $\mathcal{Z}_f$ je omejena mnozica.
            \end{definicija}

            \begin{trditev}
                Naj bo $(\Omega, \F)$ merljiv prostor in $f:\Omega\rightarrow [0, \infty]$. $f$ je $\F$-enostavna $\iff$
                $f = \sum_{i = 1}^{n}c_i\mathds{1}_{A_i}$ za neke $c_i \in [0, \infty), A_i \in \F, i \in [n]$ za nek $n \in \N$. Naprej, ce je $f \in \F\backslash\mathcal{B}_{[-\infty, \infty]},$
                potem je $\min\left(2^{-n}\lfloor2^nf\rfloor, n\right)_{n \in \N}$ zaporedje $\F$-enostavnih funkcij, ki narascajo proti ($\uparrow$) $f$. (Celo enakomerno na vsaki mnozici na kateri je $f$ omejena).
            \end{trditev}

            \begin{proof}
                $(\Rightarrow):$ $f = \sum_{\mathcal{Z}_f\backslash\{0\}}a\cdot\mathds{1}_{\{f = a\}}, \{f = a\}$ pomeni $f^{-1}(\{a\}) \in \F.$
                $(\Leftarrow):$ Baje da je ocitno (V skripti ni dokaza).
            \end{proof}

            \begin{opomba}
                \begin{enumerate}
                    \item $f \in \F\backslash\mathcal{B}_{[-\infty, \infty]} \iff f = 1 - \lim\F$-enostavnih funkcij.
                    \item $f \in \F\backslash\mathcal{B}_{[-\infty, \infty]} \iff f =$ limita linearnih kombinacij indikatorjev mnozic iz $\F$.
                \end{enumerate}

            \end{opomba}

            \begin{posledica}(izrek o monotonem razredu)
                Naj bo $\F$ $\sigma$-algebra in $\mathcal{M} \subset \F\backslash\mathcal{B}_{[0, \infty]}.$ Ce velja
                \begin{enumerate}
                    \item $\mathds{1}_A \in \mathcal{M} \ \text{za} \ \forall A \in \F$
                    \item $\mathcal{M}$ je konveksen stozec; tj. $af + g \in \mathcal{M} \ \text{za} \ \forall a \in [0, \infty] \ \forall f \in \mathcal{M} \ \forall g \in \mathcal{M}$
                    \item $\mathcal{M}$ zaprt za $\uparrow$ limite; tj. $\lim_{n \rightarrow \infty}f_n \in \mathcal{M} \ \forall$ narascajoce zaporedje $(f_n)_{n \in \N} $ v $\mathcal{M}$
                \end{enumerate}
                Potem je $\mathcal{M} =  \F\backslash\mathcal{B}_{[0, \infty]}.$
            \end{posledica}

            \begin{proof}
                Po 1. in 2. $\mathcal{M}$ vsebuje vse $\F$-enostavne funkcije. Vsaka funckija iz $\F\backslash\mathcal{B}_{[0, \infty]}$ je po trditvi 1.48
                $\uparrow$-limita $\F$-enostvnih funkcij in zato pripada $\mathcal{M}$ po 3.
            \end{proof}

            \begin{opomba}
                Pomembnost tega rezultata postane jasna v nadaljevanju, 
                saj nam pove  da "razsirimo" \ trditev ki velja za indikatorje 
                merljivih mnozic na vse nenegativne preslikave (z uporabo linearnosti 
                in monotone konvergence) in nato na vse merljive numericne preslikave 
                (po linearonosti in zapisu funkcije kot vsoto njenega negativnega in 
                pozitivnega dela).
            \end{opomba}

            \begin{trditev}(Doob-Dynkinova faktorizacijska lema)
                Naj bo $X:\Omega \rightarrow A$ in  $(A, \A)$ merljiv prostor. Potem je 
                $$
                Y \in \sigma^{\A}(X)\backslash\mathcal{B}_{[-\infty, \infty]} \iff \left(\exists h \in \A\backslash\mathcal{B}_{[-\infty, \infty]}, \ da \ je\  Y = \underbrace{h \circ X}_{h(X)}\right).
                $$
            \end{trditev}

            \begin{proof}
                $(\Leftarrow):$ $X \in X^{-1}(\A)\backslash\A$ in $h \in \A\backslash\mathcal{B}_{[-\infty, \infty]} \Rightarrow$ (kompozitumi merljivih preslikav so merljive) $\Rightarrow h \circ X \in X^{-1}(\A)\backslash\mathcal{B}_{[-\infty, \infty]}.$

                $(\Rightarrow):$ Naj bo $\mathcal{M}:= \{Y \in \sigma^{\A}(X)\backslash\mathcal{B}_{[0, \infty]}\mid\exists h \in \A\backslash\mathcal{B}_{[0, \infty]} \ \text{da je} \ Y= h(X)\}$ Potem je 
                $\mathcal{M}$ konvesken stozec zaprt za $\uparrow$ limite in po definiciji $\sigma^{\A}(X)$ vsebuje vse indikatorje mnozic iz $\sigma^{\A}(X)$, torej je po izreku o monotonem razredu
                $\mathcal{M}= \sigma^{\A}(X)\backslash\mathcal{B}_{[0, \infty]}.$ Torej ce je $Y\in \sigma^{\A}(X)\backslash\mathcal{B}_{[-\infty, \infty]}$
                lahko za $h_+$ in $h_-$ iz $\A\backslash\mathcal{B}_{[0, \infty]}$ zapisemo $Y = Y^+ - Y^-$ kjer sta $Y^+ = h_+(X)$ in $Y^- = h_-(X)$. Velja $h_+ - h_- \in \A\backslash\mathcal{B}_{[-\infty, \infty]}$.
            \end{proof}


            \begin{definicija}(Dynkinov sistem)
                Naj bo $\mathcal{D} \subset 2^\Omega.$ $\mathcal{D}$ je Dynkinov (tudi $\lambda$-) sistem na $\Omega \iff \Omega \in \mathcal{D}$ in $(B\backslash A \in \mathcal{D}\ \text{za} \ \mathcal{D} \ni A \subset B \in \mathcal{D})$ in
                za vsako zaporedje $(A_i)_{i\in\N}$ v  $\mathcal{D}$ za katerega velja $A_i \subset A_{i+1}$ za $\forall i \in \N$ velja $\cup_{i\in\N}A_i \in \mathcal{D}$.
            \end{definicija}

            \begin{definicija}
                $\mathcal{D}$ je $\pi$-sistem $\iff \mathcal{D}$ je se dodatno zaprta za $\cap$.
            \end{definicija}

            \begin{zgled}
                $\{(-\infty, a]: a \in \R\}$ je $\pi$-sistem.
            \end{zgled}

            \begin{trditev}
                Naj bo $\mathcal{D} \subset 2^{\Omega}.$ $\mathcal{D}$ je Dynkinov sistem na $\Omega \iff \Omega \in \mathcal{D}$, $\mathcal{D}$ je zaprt za $c^\Omega$, 
                $\cup_{i \in \N}A_i \in \mathcal{D} \ za \ \forall$ zaporedje $(A_i)_{i \in \N}$ v $\mathcal{D}$ ki ima $A_i \cap A_j = \emptyset$ za $i \neq j$ iz $\N$. \\
                Dodatno $\mathcal{D}$ je $\sigma$-algebra na $\Omega \iff$ $\mathcal{D}$ je $\pi$-sistem in Dynkinov sistem na $\Omega$.
            \end{trditev}

            \begin{proof}
                Brez dokaza.
            \end{proof}

            \begin{definicija}
                Za $\mathcal{L} \subset 2^{\Omega}$ definiramo najmanjsi Dynkinov sistem na $\Omega$ ki vsebuje $\mathcal{L}$ kot podmnozico
                $$
                    \lambda_{\Omega}(\mathcal{L}):= \cap\{\mathcal{D} \in 2^{2^{\Omega}} \mid \mathcal{D} \ \text{ je Dynkinov sistem na $\Omega$ in $\mathcal{L} \subset \mathcal{D}$}\}.
                $$
            \end{definicija}

            \begin{trditev}
                Naj bo $\mathcal{L}$ $\pi$-sistem in $\mathcal{L}\subset 2^{\Omega}.$ Potem je $\lambda_{\Omega}(\mathcal{L}) = \sigma_{\Omega}(\mathcal{L})$.
            \end{trditev}

            \begin{proof}
                Brez dokaza.
            \end{proof}

            \begin{posledica}(Dynkinova lema)
                Naj bo $\mathcal{L}$ $\pi$-sistem in $\mathcal{D}$ Dynkinov sistem na $\Omega$ in $\mathcal{L} \subset \mathcal{D}.$ Potem je $\sigma_{\Omega}(\mathcal{L}) \subset \mathcal{D}.$
                
            \end{posledica}

            \begin{proof}
                Brez dokaza.
            \end{proof}

            Pomembnost zgoraj navedenega rezultata izhaja predvsem iz naslednje opazke. 
            Naj bo $\A$ $\sigma$-algebra in $\mathcal{L}\subset 2^{\Omega}$ $\pi$-sistem. Predpostavimo, da je $\A = \sigma_{\Omega}(\mathcal{L})$ 
            Pogosto v teoriji mere (in zlasti v teoriji verjetnosti) želimo pokazati,
             da neka lastnost, $P(A) \in \A$, velja za vse elemente 
             $A\in\A$, in to želimo storiti, potem ko smo že prejeli ali že predhodno 
             vzpostavili, da $P(A)$ velja za vse $A\in\mathcal{L}$. Običajno je enostavno 
             (ker se mere lepo obnašajo pri disjunktnih/neopadajočih števnih unijah 
             in ker se končne mere tudi lepo obnašajo pri "primerljivih razlikah"/komplementih) 
             neposredno preveriti, da je zbirka $\mathcal{D}:= \{A\in\sigma_{\Omega}(\mathcal{L})\mid P(A)\}$ $\lambda$-sistem, vendar pa 
             običajno ni tako enostavno (ker se mere ne obnašajo tako lepo pri poljubnih števnih 
             unijah), preveriti neposredno, da je $\mathcal{D}$ $\sigma$-algebra. Lema $\pi\backslash\lambda$ vzpostavi izjemno uporabno 
             bližnjico za to posredno preverjanje. Njena uporabnost je še dodatno okrepljena s tem, 
             da običajno ni težko najti generirajočega $\pi$-sistema, na katerem lastnost $P$ "očitno" velja.
              Naslednji rezultat je dosežen v skladu z zgoraj navedenim in ga bomo še večkrat uporabili
             v preostanku tega besedila.

            \begin{trditev}(Meri, ki se strinjata na $\sigma$-lokalizirajocem generirajocem $\pi$-sistemu)
                Naj bosta $\mu$ in $\nu$ meri na merljivem prostoru $(E, \Sigma)$, naj bo $\mathcal{L} \subset \Sigma$ $\pi$-sistem za katerega
                velja $\sigma_{E}(\mathcal{L}) = \Sigma$. Naj velja $\mu|_\mathcal{L} = \nu|_\mathcal{L}$ in naj obstaja 
                zaporedje $(L_n)_{n\in\N}$ v $\mathcal{L}$, ki ga sestavljajo $\uparrow$ paroma disjunktne mnozice,
                da velja $\mu(L_n) = \nu(L_n) < \infty$ za vsak $n\in\N$ in $\cup_{n\in\N}L_n = E.$ Potem $\mu = \nu.$
            \end{trditev}

            \begin{proof}
                Denimo, da smo trditev pokazali v primeru, ko sta $\mu$ in $\nu$ koncni
                meri z enako maso. Potem za vsak $n \in \N$ velja, da sta $\mu|_{L_n}$ in
                $\nu|_{L_n}$ koncni meri z enako maso. (Namrec $\mu(L_n) = \nu(L_n) < \infty$ na merljivem 
                prostoru $(L_n, \Sigma|_{L_n})$). $\mathcal{L}|_{L_n}$ je $\pi$-sistem ([$= \mathcal{L} \cap 2^{L_n}, \Rightarrow L_n \in \mathcal{L}$] in $\sigma_{L_n}(\mathcal{L}|_{L_n}) = \sigma_E(\mathcal{L})|_{L_n} = \Sigma|_{L_n}$)
                in $(L_n)_{n\in\N}$ je zaporedje v $\mathcal{L}|_{L_n}$ za katerega je res 
                $\mu|_{L_n}(L_n) = \nu|_{L_n}(L_n)<\infty$ ter $\cup_{n\in\N}L_n=Ln.$
                Dokoncaj dokaz.
            \end{proof}

            Z besedami: dve meri ki se strinjata na $\sigma$-lokalizirajocem generirajocem $\pi$-sistemu se strinjata povsod.

        \subsubsection{Lebesque-Stieltjesove mere}
            \begin{izrek}(Lebesque-Stieltjesove mere)
                Naj bo $F:\R \rightarrow \R$ nepadajoca in zvezna z desne. Potem obstaja natanko ena mera $\mu$ na $\mathcal{B}_\R$ za katero velja
                $$
                    \mu((a, b]) = F(b) - F(a) \ \text{za} \ a, b \in \R, a < b.
                $$
            \end{izrek}

            \begin{proof}(Enolicnosti)
                $\{(a, b]\mid a\leq b \in \R\}$ je $\pi$ sistem, ki generira $\mathcal{B}_\R$ na $\R$
                in je tudi lokalizirajoc za $\mu:\cup_{n\in\N}(n, n+1] = \R$. Apliciramo prejsnjo trditev o 
                strinjanju mer na $\sigma$-lokalizirajocem generirajocem $\pi$-sistemu. (Ce sta meri 
                $\mu$ in $\nu$ verjetnostni, lahko pogoj, ki se tice obstoja $(L_n)_{n\in\N}$ izpustimo, ker 
                lahko zamenjamo $\mathcal{L}$ z $\mathcal{L} \cup \{E\}$ in vzamemo $L_n = E  \ \forall n \in \N$).
            \end{proof}

            \begin{definicija}
                Meri $\mu$ iz izreka pravimo Lebesque-Stieltjesova mera prirejena $F$ in jo oznacimo z $dF$.
                ($\mathcal{L}:= d(id_\R)$ je Lebesqeova mera na $\R$, ne obstaja razsiritev $\mathcal{L}$ na mero na $(\R, 2^\R)$).
            \end{definicija}

            \begin{trditev}
                Naj bo $F:\R \rightarrow \R$ nepadajoca in zvezna z desne. Mera $dF$ je: $\sigma$-koncna; koncna $\iff$ $F$ je omejena;
                verjetnostna $\iff$ $\lim_{\infty}F - \lim_{-\infty}F = 1$ Za $x \in \R$ je $dF(\{x\}) = F(x) - \underbrace{F(x-)}_{\text{leva limita $F$ v $x$}}$. 

            \end{trditev}
            \begin{proof}
                $\sigma$-koncnost: \\
                $\cup_{n \in \mathbb{Z}}(n, n+1]=\R$; \ 
                $dF((n, n+1]) = F(n + 1) - F(n) < \infty \ \forall n \in \mathbb{Z}$; \ $dF((-n, n]) = F(n) - F(-n)$ \ $\Rightarrow$ ko gre $n \rightarrow \infty$ $\Rightarrow$ $dF(\R) = \lim_\infty F = \lim_{-\infty}F$.
                Od tod dobimo karakterizacijo koncnosti $dF$ in kdaj je $dF$ verjetnostna mera.
                Za $x \in \R$ je $(x - \frac{1}{n}, x] \downarrow \{x\}$, ko gre $n \rightarrow \infty$ cez $\N$ in zato je po zveznosti
                $dF$ od zgoraj na mnozicah s koncno mero $dF(\{x\}) = \lim_{n \rightarrow \infty}dF((x - \frac{1}{n}, x]) = \lim_{n \rightarrow \infty}F(x) - F(x - \frac{1}{n})$.
            \end{proof}

            \begin{zgled}
                $\mathcal{L}$ je $\sigma$-koncna in $\mathcal{L}(A) = 0$ za vsako stevno $A\subset\R$.
            \end{zgled}

            \begin{zgled}(Cantorjeva mnozica)
                Naj bo $V$ preslikava na intervalih oblike $[0, 1]$. $V(A)$ odstrani srednjo tretjino vsakega intervala iz $A$.
                Torej $V([0, 1]) = [0, \frac{1}{3}] \cup [\frac{2}{3}, 1]$. Potem je $C:= \cap_{n\in\N}V^n([0, 1])$ nestevna kompaktna 
                mnozica z $\mathcal{L}(C) = 0$.
            \end{zgled}

    \subsection{Integracija na merljivih prostorih}
            Uvod v abstraktno integracijo.
            Naj bo $\mathcal{P}$ particija $\Omega$ in $\mu':\mathcal{P} \rightarrow [0, \infty)$ in
            $f':\mathcal{P} \rightarrow [0, \infty)$ ($\mu$ je mera na $\sigma_\Omega(\mathcal{P})=\{\cup Q: Q \in 2^{\mathcal{P}}\}; \ 
            \mu(\cup Q):= \sum_{p \in Q}\mu'(p)$ za $Q \subset \mathcal{P}$), ($f:\Omega \rightarrow [0, \infty); f(\omega) = f'(p)$ za $\omega \in p \in \mathcal{P}$)
            Potem je $\sum_{p \in \mathcal{P}}f'(p)\cdot\mu'(p) = \sum_{r \in \mathcal{Z}_f}r\cdot\mu(\underbrace{\{f = r\}}_{f^{-1}(\{r\})})$.
            $\sum_{p \in \mathcal{P}}f'(p)\mu'(p) = \int_a^bf(z)dz$ Riemann - Darbouxov integral odsekoma konstantne funckije 
            $f: [a, b] \rightarrow [0, \infty)$.

        \subsubsection{Lebesqueov integral}
            \begin{definicija}
                Naj bo $(\Omega, \F, \mu)$ prostor z mero in $f \in \F\backslash\mathcal{B}_{[-\infty, \infty]}$
                \begin{itemize}
                    \item Ce je $f$ $\F$-enostvna, potem je
                    $$
                        \int fd\mu := \sum_{a \in \mathcal{Z}_f}a\cdot\mu(\underbrace{\{f = a\}}_{f^{-1}(\{a\})})
                    $$
                    \item Ce $f$ ni $\F$-enostvna, je pa $f \geq 0$, potem je
                    $$
                        \int fd\mu := \sup\{\int qd\mu: \text{q je $\F$-enostvna in $q  \leq f$}\}
                    $$
                    \item Ce f ni $\geq 0$, potem je 
                    $$
                        \int fd\mu := \int f^+d\mu - \int f^-d\mu
                    $$
                \end{itemize}
                $\int fd\mu$ pravimo integral $f$ proti $\mu$ (tudi pricakovana vrednost, ce je $\mu$ verjetnostna mera).
                Druge notacije za $\int fd\mu$ so $\mu[f]:=\mu^x[f(x)]:=\int f(x)\mu(dx).$
            \end{definicija}

            Za $A \in \F$ pisemo $\mu[f;A]:= \mu^x[f(x);x \in A]:= \int_A f(x)\mu(dx):= \int f\mathds{1}_Ad\mu$.

            \begin{definicija}
                Integral $f$ proti $\mu$ je dobro definiran $\iff$ $\int f^+d\mu \wedge\int f^-d\mu < \infty$.
                $f$ je $\mu$-integrabilen $\iff $ $\int f^+d\mu \vee \int f^-d\mu < \infty$.
            \end{definicija}

            \begin{definicija}
                Naj bo $(\Omega, \F, \mu)$ prostor z mero. 
                $$
                    \mathcal{L}^1(\mu) := \{f \in \F\backslash\mathcal{B}_{\R}: \text{$f$ je $\mu$ integrabilna}\}. 
                $$
                Za $g:\Omega \rightarrow \mathbb{C}$ za katero je $\{Re(g), Im(g)\} \subset \mathcal{L}^1(\mu)$ je 
                $\int g d\mu:= \int Re(g)d\mu + i\int Im(g)d\mu$.
            \end{definicija}

            \begin{izrek}
                Naj bo $(\Omega, \F, \mu)$ prostor z mero. Integral ima sledece lastnosti.
                \begin{enumerate}
                    \item Aditivnost: $\int (f + g)d\mu = \int fd\mu + \int gd\mu$ za $\{f, g\} \subset \F\backslash\mathcal{B}_{[-\infty, \infty]}$ take, da je $\int f^{-1}d\mu \vee \int g^{-1}d\mu < \infty$.
                    \item Integral indikatorja: $\int \mathds{1}_Ad\mu = \mu(A)$ za $\forall A \in \F$. ( V posebnem je $\int 0 d\mu = 0$ in torej $\int fd\mu = \int f^{+}d\mu - \int f^{-}d\mu$ za vse $f \in \F\backslash\mathcal{B}_{[-\infty, \infty]}$)
                    \item Integrali, ki so nic, ki so koncni: Za $f \in \F\backslash\mathcal{B}_{[0, \infty]}$ je $\int f d\mu = 0 \iff \mu(f >0)= 0$ ($f$ je skoraj povsod glede na $\mu$), Ce $\int f d\mu < \infty \Rightarrow \mu(f = \infty) = 0$ ($f < \infty$ skoraj povsod glede na $\mu$.)
                    \item Trikotniska neenakost: $\left|\int fd\mu\right| \leq \int |f|d\mu$ za $\forall f \in \F\backslash\mathcal{B}_{[-\infty, \infty]}$.
                    \item Integral ne vidi mnozic z mero 0: Ce je $\int fd\mu = \int gd\mu$ za $\{f, g\} \subset \F\backslash\mathcal{B}_{[-\infty, \infty]}$ za katere je $f = g$ skoraj povsod glede na $\mu$ ($\mu(f \neq g) = 0$)
                    \item Monotonost: $\int gd\mu \leq \int fd\mu$, brz ko je $\{g, f\} \subset \F\backslash\mathcal{B}_{[-\infty, \infty]}$, $g\leq f$ in $\int gd\mu < \infty$.
                    \item Homogenost: $\int cfd\mu = c\int fd\mu$ za vsak $f \in \F\backslash\mathcal{B}_{[-\infty, \infty]}$ za katero velja $\int(cf)^-d\mu \wedge \int(cf)^+d\mu < \infty$ za vsak $c \in (-\infty, \infty)$.    
                    
                
                \end{enumerate}
            \end{izrek}

            Vsi integrali ki zadostutujejo 1.-4. so dobro definirani. Enako velja za 7. le v primeru, ko je $c = 0$ in $\mu[f^+] = \mu[f^-] = \infty$. Za 5. je $\int fd\mu$ dobro definiran $\iff$ $\int gd\mu$ je dobro definiran.

            \begin{trditev}
                Naj bosta $a \leq b \in \R$ in $f: [a, b] \rightarrow \R$ zvezna. Potem
                je $f\mathds{1}_{[a, b]}$ $\mathcal{L}$-integrabilna in 
                $$
                    \int_{[a, b]}fd\mathcal{L} = \int_a^bf(x)dx,
                $$
                kjer je na desni Riemann-Darbouxov integral.
            \end{trditev}

            \begin{zgled}
                $\mathds{1}_{\mathbb{Q} \cap [0, 1]}$ je $\mathcal{L}$-integrabilna (ker je stevna $\equiv$ 0), ni pa R-D integrabilna.
            \end{zgled}
        \subsubsection{Konvergencni izreki s posledicami}
            \begin{izrek}
                Naj bo $(\Omega, \F, \mu)$ prostor z mero in $(f_n)_{n\in\N}$ zaporedje v $\F\backslash\mathcal{B}_{[-\infty, \infty]}$.
                \begin{enumerate}
                    \item Naj obstaja $g \in \F\backslash\mathcal{B}_{[0, \infty]}$, $\int gd\mu < \infty$, za katero je $g \geq f_n^-$ za vsak $n\in\N.$
                    \begin{enumerate}
                        \item Povezanost od spodaj (Fatoujeva lema):
                            $$
                                \mu\left[(\liminf_{n\rightarrow \infty}f_n)^-\right] < \infty \ \text{in} \
                                \int \liminf_{n\rightarrow \infty}f_n d\mu \leq \liminf_{n\rightarrow \infty}\int f_n d\mu.
                            $$
                        \item Monotona konvergenca (Lévi):
                            Ce je $f_n \leq f_{n+1}$ za $\forall n \in \N$, potem je:
                            $$
                                \mu\left[(\lim_{n\rightarrow \infty}f_n)^-\right] < \infty \ in \ \int\lim_{n\rightarrow\infty}f_nd\mu = \uparrow-\lim_{n\rightarrow\infty}\int f_nd\mu.
                            $$
                    \end{enumerate}
                    \item Dominirana konvergenca (Lebesque):
                        Ce obstaja $g \in \F\backslash\mathcal{B}_{[0, \infty]}$, ki je $\mu$-ntegrabilna in za katero je $|f_n| \leq g$ za $\forall n \in \N$ in ce obstaja $\lim_{n\rightarrow \infty}f_n$(povsod), potem je:
                        $$
                            \mu\left[|\lim_{m\rightarrow\infty}f_m|\right] < \infty \ in \ \lim_{n\rightarrow\infty}\int|f_n - \lim_{m\rightarrow\infty}f_m|d\mu = 0 \ in \ torej \ \int\lim_{n\rightarrow\infty}f_nd\mu = \lim_{n\rightarrow\infty}\int f_nd\mu.
                        $$
                \end{enumerate}
            \end{izrek}
            \begin{proof}
                (ib) $\Rightarrow$ (ia) in (ii):
                (ia):
                $$
                    \left(inf_{n \geq m} f_n\right)_{m \in \N} je zaporedje v \F\backslash\mathcal{B}_{[-\infty, \infty]},
                $$
                ki je narascajoce. $f_n^- \leq g \Rightarrow \left(inf_{n \geq m} f_n\right)^- \leq g$. Po (ib) sledi da 
                $$
                    \int \lim_{m \rightarrow \infty}f_m d\mu =  \int \lim_{m \rightarrow \infty} inf_{n \in \N \geq m} f_n d\mu
                    = lim_{m \rightarrow \infty} \int inf_{n \in \N \geq m} fn d\mu \leq liminf_{n \rightarrow \infty} \int f_n d\mu.
                $$
                (ii):
                Uporabim (ia). Za $\left(- |\underbrace{fn - lim_{m\rightarrow \infty} f_m}_{\leq 2g}|\right)_{n \in \N}$ in dobimo 
                $$
                    0 = \int liminf_{n \rightarrow \infty} \left(- |\underbrace{fn - lim_{m\rightarrow \infty} f_m}_{\leq 2g}|\right) d\mu
                    \leq ...
                $$
            \end{proof}

            \begin{posledica}
                Naj bo $(\Omega, \F, \mu)$ prostor z mero in $(f_n)_{n \in \N}$ zaporedje v $\F\backslash\mathcal{B}_{[0, \infty]}$.
                Potem je 
                $$
                    \int \sum_{n \in \N_0}f_n d\mu = \sum_{n \in \N_0}\int f_n d\mu.
                $$
            \end{posledica}

            \begin{proof}
                $$
                    \sum_{n \in \N_0}f_n = \lim_{n \rightarrow \infty}\sum_{m = 1}^n f_m \ (narascajoce \ v \ n)
                $$
                Zaradi prejsnjega izreka sledi
                $$
                    \int \sum_{n \in \N_0}f_n d\mu = \lim_{n \rightarrow \infty}\int \sum_{m = 1}^n f_m d\mu = 
                    \lim_{n \rightarrow \infty}\sum_{m = 1}^n \int f_m d\mu = \sum_{n \in \N_0}\int f_n d\mu.
                $$
            \end{proof}
            
            \begin{posledica}
                Naj bo $(\Omega, \F)$ merljiv prostor in $(\mu_n)_{n \in \N}$ zaporedje mer na $(\Omega, \F)$. Potem je 
                $\sum_{n \in \N}\mu_n$ mera na $(\Omega, \F)$ in za $f \in \F\backslash\mathcal{B}_{[-\infty, \infty]}$ je
                $$
                    \int fd\left(\sum_{n \in \N}\mu_n\right) \ dobro \ definirana \ \iff 
                    \left( \sum_{n \in \N}\int f^+d\mu_n\right) \wedge \left(\sum_{n \in \N}\int f^-d\mu_n\right) < \infty.
                $$
                V tem primeru je 
                $$
                    \int f d(\sum_{n \in \N}\mu_n) = \sum_{n \in \N}\int fd\mu_n.
                $$
            \end{posledica}

            \begin{proof}
                Za $A \in \F$ je $\sum_{n \in \N}\mu_n(A) = \int \mu_n(A)c_\N(dn)$. Potem je za zaporedje $(A_m)_{m \in \N}$
                paroma disjunktnih mnozic v $\F$:
                \begin{align*}
                    \left( \sum_{n \in \N}\mu_n \right)\left(\cup_{m \in \N}A_m\right) &= \int \mu_n \left(\cup_{m \in \N}A_m\right)c_\N(dn) \\
                    &= \int \sum_{m \in \N}\mu_n(A_m)c_\N(dn) = \\
                    &= \sum_{m \in \N} \int \mu_n(A_m)c_\N(dn) = \\
                    &= \sum_{m \in \N}\left(\sum_{n\in \N}(\mu_n(A_m))\right) = \\
                    &= \sum_{m \in \N}\left[\left(\sum_{n\in\N}\mu_n\right)(A_m)\right]                    
                \end{align*}
                Poleg tega je
                $\left(\sum_{n\in\N}\mu_n\right)(\emptyset) = \sum_{n\in\N}\mu_n(\emptyset) = \sum_{n\in\N}0 = 0.$
                Torej je $\sum_{n\in\N}\mu_n$ mera na $(\Omega, \F)$.
            \end{proof}

            Naprej, razred funkcij $f \in \F\backslash\mathcal{B}_{[-\infty, \infty]}$ za katere velja prejsnja posledica je
            $$
                \int f d\left(\sum_{n\in\N}\mu_n\right) = \int \int f d\mu_n c_\N(dm)
            $$
            konveksen stozec zaprt za $\uparrow$ limite, ki svebuje $\mathds{1}_A$ za $A \in \F$.
            Po izreku o monotonem razredu prejsnja posledica velja za vse $f \in \F\backslash\mathcal{B}_{[-0, \infty]}$.

            \begin{definicija}
                Naj bo $(\Omega, \F, \mu)$ prostor z mero in $(\Omega', \F')$ merljiv prostor. Za $ \in \F\backslash\F'$ vpeljemo
                $$
                    f * \mu: \F' \rightarrow [0, \infty], \
                    (f*\mu)(A') := \mu(f^{-1}(A')) \ \text{za $A'\in \F'$}.
                $$
                $f*\mu$ oznacimo tudi z $\mu \circ f^{-1}$ ali $\mu_{f}$ in ji recemo potisk $\mu$ po $f$ glede na $\F'$, ali 
                zakon $f$ pod $\mu$ glede na $\F'$ (v primeru, da je verjetnostna).
            \end{definicija}

            \begin{definicija}
                Z $f*\mu$ oznacimo sliko mere $\mu$ pod $f$ glede na $\F$.
            \end{definicija}

            \begin{posledica}
                Naj bo $(\Omega, \F, \mu)$ prostor z mero in $(\Omega', \F')$ merljiv prostor. Za $f \in \F\backslash\F'$ je
                $f*\mu$ mera na $(\Omega', \F')$, ki je koncna ali verjetnostna - kakor je $\mu$. Poleg tega je za $g \in \F'\backslash\mathcal{B}_{[-\infty, \infty]}$
                $$
                    \int g d(f*\mu) \ \text{dobro definiran} \iff \int g \circ f d\mu \ \text{dobro definiran}.
                $$
                in tedaj je 
                $$
                    \int g d(f*\mu) = \int g \circ f d\mu.
                $$
            \end{posledica}

            \begin{proof}
                $(f*\mu)(\emptyset) = \mu(f^{-1}(\emptyset)) = \mu(\emptyset) = 0.$
                Za zaporedje $(A_n)_{n \in \N}$ paroma disjunktnih mnozic v $\F'$ je 
                $(f*\mu)(\cup_{n\in\N}A_n) = \mu(f^{-1}(\cup_{n \in \N}A_n)) = \mu(\cup_{n\in\N}f^{-1}(A_n)) = \sum_{n\in\N}\mu(f^{-1}(A_n)) = \sum_{n\in\N}(f*\mu)(A_n).$	
                Torej velja da je $f*\mu$ mera na $(\Omega', \F')$.
                Naprej, $(f*\mu)(\Omega') = \mu(f^{-1}(\Omega')) = \mu(\Omega),$ torej je res $f*\mu$ koncna oz. verjetnostna
                $\iff$ je $\mu$ koncna oz.\ verjetnostna.
            \end{proof}

            Postavimo sedaj
            $$
            \mathcal{M} := \{g \in \F'\backslash\mathcal{B}_{[0, \infty]} \mid \int gd(f* \mu) = \int g \circ f d\mu\}.
            $$
            $\mathds{1}_A \in \mathcal{M} \ \forall A \in \F',$ kajti $\int \mathds{1}_Ad(f*\mu) = (f*\mu)(A) = \mu(f^{-1}(A)) = \int\mathds{1}_{f^{-1}(A)}d\mu$. \\
            $\mathcal{M}$ je konveksen stozec.
            \begin{proof}
                Za $\forall a \in [0, \infty)$ \ $\forall q_1, q_2 \in \mathcal{M}$ je $aq_1 + q_2 \in \F'\backslash\mathcal{B}_{[0, \infty]}$, in
                $\int (aq_1 = q_2)d(f*\mu) = a\int q_1d(f*\mu) + \int q_2d(f*\mu) = a\int q_1 \circ fd\mu + \int q_2 \circ fd\mu = \int (aq_1 + q_2)\circ fd\mu$. 
                Torej je $aq_1 + q_2 \in \mathcal{M}$.
            \end{proof}

            $\mathcal{M}$ je zaprta za $\uparrow$ limite.
            \begin{proof}
                Za vsako $\uparrow$ zaporedje $(q_n)_{n \in \N}$ v $\mathcal{M}$ je $\lim_{n \rightarrow \infty}q_n \in \F'\backslash\mathcal{B}_{[0, \infty]}$ in
                $\int \lim_{n\rightarrow\infty}q_nd(f*\mu) = \lim_{n\rightarrow\infty}\int q_nd(f*\mu) = \lim_{n\rightarrow\infty}\int q_n\circ f d\mu = \int \lim_{n\rightarrow\infty}(q_n\circ f)d\mu = \int(\lim_{n\rightarrow\infty}q_n)\circ fd\mu$.
                Torej je $\lim_{n\rightarrow\infty}q_n \in \F'\backslash\mathcal{B}_{[0, \infty]}$.
            \end{proof}

            Za splosen $g \in \F'\backslash\mathcal{B}_{[-\infty, \infty]}$ je $\{g^+, g^-\} \subset \F'\backslash\mathcal{B}_{[-\infty, \infty]}$,
            zato $\int g^+d(f* \mu) = \int g^+ \circ $ ... to je del dokaza.

            \begin{posledica}
                Naj bo $(X, \Sigma, \mu)$ prostor z mero in $O$ neka odprta podmnozica $\R$ in $f:X \times O \rightarrow \R$. Denimo, da je 
                \begin{itemize}
                    \item $F(., t)$ $\mu$-integrabilna za $\forall t \in O$
                    \item $F(x, .)$ odvedljvia za $\forall x \in X$ 
                \end{itemize}
                Predpostavimo, da $\exists$ $g \in \Sigma\backslash\mathcal{B}_{[0, \infty]}, \int g d\mu < \infty,$ taka da je 
                $$
                    \left|\frac{\partial F}{\partial t}(x, t) \right| \leq g(x) \ \forall x \in X \ \forall t \in O.
                $$
                Potem velja 
                \begin{enumerate}
                    \item $(X \ni x \rightarrow \frac{\partial F}{\partial t}(x, t))$ je $\mu$-integrabilna
                    \item $(O \ni t \rightarrow \int F(x,t)\mu(dx))$ je odvedljiva in 
                    \item $\frac{d}{dt}\int F(x, t)\mu(dx) = \int \frac{\partial F}{\partial t}(x, t)\mu(dx). \ \forall t \in O$
                \end{enumerate}
            \end{posledica}

            \begin{proof}
                Dokaz v skripti (stran 19).
            \end{proof}

        \subsubsection{Zamenjava vrstnega reda integracije in povezane teme}
            
            \begin{definicija}
                Naj bosta $(\Omega, \F)$ in $(\Omega', \F')$ merljiva prostora. 
                $$
                    \F \otimes \F' := \sigma_{\Omega \times \Omega'}(\{A \times A' \mid (A, A') in \F \times \F'\})
                $$
                je produktna $\sigma$-algebra $\F$ in $\F'$ $(\mathcal{B}_{\R^n} := \mathcal{B}_{\R} \otimes \cdots \otimes \mathcal{B}_{\R})$ je Borelova
                $\sigma$-algebra na $\R^n$.
            \end{definicija}

            \begin{trditev}
                Za $A \subset \R^2$ in $f:A \rightarrow [-\infty, \infty]$, ki je zvezna je $f \in \mathcal{B}_A\backslash\mathcal{B}_{[-\infty, \infty]}$.
            \end{trditev}

            \begin{proof}
                Brez dokaza.
            \end{proof}

            \begin{trditev}
                Naj bosta $(\Omega, \F)$ in $(\Omega', \F')$ merljiva prostora. Potem velja:
                \begin{enumerate}
                    \item $\F \otimes \F' $ je najmanjsa $\sigma$-algebra $\mathcal{G}$ na $\Omega \times \Omega$ za katero je \\
                     $pr_\Omega := (\Omega \times \Omega' \ni (\omega, \omega') \rightarrow w) \in \mathcal{G}\backslash\F$ in \\
                      $pr_{\Omega'} := (\Omega \times \Omega' \ni (\omega, \omega') \rightarrow \omega') \in \mathcal{G}\backslash\F'$.
                    \item Za $f \in (\F \otimes \F')\backslash\mathcal{B}_{[-\infty, \infty]} \Rightarrow$ $f(\omega, .) \in \F'\backslash\mathcal{B}_{[-\infty, \infty]} \ \forall \omega \in \Omega$ in $f(., \omega') \in \F\backslash\mathcal{B}_{[-\infty, \infty]} \ \forall \omega' \in \Omega'$.
                    \item Za vsako $\sigma$-algebro $\mathcal{G}$ na $G$ in $f: G \rightarrow \Omega$ ter $f':G \rightarrow \Omega'$ je $(f, f') \in \mathcal{G}\backslash\F\otimes\F' \iff f \in \mathcal{G}\backslash\F$ in $f'\in \mathcal{G}\backslash\F'.$
                \end{enumerate}
            \end{trditev}

            \begin{proof}
                Brez dokaza.
            \end{proof}

            \begin{izrek}
                Naj bosta $(\Omega, \F, \mu)$ in $(\Omega', \F', \mu')$ prostora z mero ter $\mu$ in $\mu'$ $\sigma$-koncni. Potem velja:
                \begin{enumerate}
                    \item Obstaja natanko ena mera $\nu$ na $\F \otimes \F'$, ki jo oznacimo $\mu \times \mu'$, da je $\nu(A \times A') = \mu(A)\mu'(A') \ \forall (A, A') \in \F \times \F'$.
                    \item Naj bo $f \in (\F \otimes \F')\backslash\mathcal{B}_{[-\infty, \infty]}$. in naj velja:
                        \begin{enumerate}
                            \item $ f \geq 0$ (Tonelli) ali
                            \item $\int|f|d(\mu \times \mu') < \infty$ (Fubini) ali
                            \item $\iint f^-(\omega, \omega')\mu(d\omega)\mu'(d\omega') \wedge \iint f^-(\omega, \omega')\mu'(d\omega')\mu(d\omega) < \infty$.
                        \end{enumerate}
                        Potem je 
                        $$
                        (\Omega \ni \omega' \rightarrow \int f(\omega, \omega')\mu(d\omega)) \in \F'\backslash\mathcal{B}_{[-\infty, \infty]}\ in \
                        (\Omega' \ni \omega \rightarrow \int f(\omega, \omega')\mu'(d\omega')) \in \F\backslash\mathcal{B}_{[-\infty, \infty]}
                        $$
                        ter
                        $$
                        \int f^-(\omega, \omega')\mu(d\omega) < \infty \ \text{skoraj povsod glede na $\mu'$ in $\omega'$} \ \text{in}
                        $$
                        $$
                        \int f^-(\omega, \omega')\mu'(d\omega') < \infty \ \text{skoraj povsod glede na $\mu$ in $\omega$}           
                        $$
                        ter
                        $$
                         \int fd(\mu\times\mu') = \iint f(\omega, \omega')\mu(d\omega)\mu'(d\omega') = \iint f(\omega, \omega')\mu'(d\omega')\mu(d\omega).
                        $$
                \end{enumerate}
            \end{izrek}

            \begin{proof}
                Brez dokaza.
            \end{proof}

            \begin{definicija}
                Meri $\mu \times \mu'$ recemo produktna mera. $(\mathcal{L}^n:= \mathcal{L}\times \cdots \times\mathcal{L}$ 
                pravimo $n$-razsezna Lebesquova mera na $(\R^n, \mathcal{B}_{\R^n})$). 
            \end{definicija}

            \begin{zgled}
                Naj bo $f:[0, \infty] \rightarrow [0, \infty]$ zvezna za katero velja $f(0) = 0$ ter naj bo na $(0, \infty)$ zvezno odvedljiva s $f'\geq 0$ na $(0, \infty)$. Naj bo $\nu$ $\sigma$-koncna mera na $\mathcal{B}_{[0, \infty]}.$ Potem je 
                \begin{align*}
                    \int f d\nu = \int \int_{0}^x f'(t)dt \nu(dx) &= \int \int_{(0, x)}f'(t)\mathcal{L}(dt)\nu(dx) = \\
                    &=\int \int f'(t)\mathds{1}_{(0, x)}(t)\mathcal{L}(dt)\nu(dx)\\
                    (Tonelli) &= \int \int f'(t)\mathds{1}_{(0, x)}(t)\nu(dx)\mathcal{L}(dt) \\
                    &= \int_{(0, \infty)}f'(t)\int \mathds{1}_{(t, \infty)}(x)\nu(dx)\mathcal{L}(dt) \\
                    &= \int_{(0, \infty)}f'(t)\nu(t, \infty)\mathcal{L}(dt). \\
                    &= \int_0^\infty f'(t)\nu(t, \infty)dt \\
                \end{align*}
                kot izlimititran Riemannov integral ce je $\nu((t, \infty)) < \infty \ \forall t\in(0, \infty)$.
            \end{zgled}

            \begin{trditev}
                Naj bo $(\Omega, \F, \mu)$ prostor z mero in $(\Omega', \F')$ merljiv prosotor. Naj bo $X \in \F\backslash\F'$. 
                Naj bo $(A, \A)$ merljiv prosotor in $D_A:= \{(a, a)\mid a\in A\} \in \A\otimes\A$ ter $\{f, g\}\subset \F'\backslash\A$.
                Potem je $f(X) = g(X)$ skoraj povosd $\iff$ $f = g$ skoraj povsod glede na $X*\mu$.
            \end{trditev}
            
            \begin{proof}
                $\{f \neq g\} = \Omega'\backslash\{f = g\} = \Omega'\backslash\left[(f, g)^{-1}(D_A)\right]$.
                $\{f, g\}\subset \F'\backslash\A$ Po trditvi 2.17 sledi $(f, g) \in \F'\backslash(\A\otimes\A)$
                $\{f \neq g\} \in \F'$
                $(X*\mu)(\{f\neq g\}) = \mu(X{-1}(\{f\neq g\})) = \mu(\{f(X)\neq g(X)\})$
            \end{proof}

        \subsubsection{Nedolocena integracija in absolutna zveznost}
            \begin{definicija}
                Naj bo $(\Omega, \F, \mu)$ prostor z mero in $f \in \F\backslash\mathcal{B}_{[-\infty, \infty]}$, 
                ki ima $\int fd\mu$ dobro definiran. Potem funkciji 
                $$
                    f \cdot \mu:= (\F \ni A \mapsto \int_A f d\mu)
                $$
                pravimo nedoloceni integral $f$ proti $\mu$.
            \end{definicija}

            \begin{definicija}
                Za dve meri $\mu$ in $\nu$ na merljivem prostoru $(\Omega, \F)$ je:
                \begin{enumerate}
                    \item $\mu << \nu \iff$ $\mu$ je absolutno zvezna glede na $\nu$ $\iff$ $\forall A \in \F$ za katerega je $\nu(A) = 0$ je tudi $\mu(A) = 0$. 
                    \item $\mu \sim \nu$ $\iff$ $\mu$ je ekvivalentna $\nu$ $\iff$ $\nu << \mu$ in $\mu << \nu.$
                \end{enumerate}
            \end{definicija}

            \begin{trditev}
                Naj bo $(\Omega, \F, \mu)$ prostor z mero in $f \in \F\backslash\mathcal{B}_{[-\infty, \infty]}$. Potem je 
                $f \cdot \mu$ mera na $\F$, ki je $<< \mu$ in 
                $$
                    \int g d(f \cdot \mu) = \int gf d\mu.
                $$
                Pri cemer je leva stran dobro definirana $\iff$ je to res za desno stran in to za $\forall g \in \F\backslash\mathcal{B}_{[-\infty, \infty]}$. V tem primeru (sociativnost nedolocene integracije) je $g \cdot (f \cdot \mu) = (gf)\cdot \mu$. Ce je $f>0$ skoraj povsod glede na $\mu$ potem je $f\cdot\mu \sim \mu$.
            \end{trditev}
            
            \begin{proof}
                Dokaz v skripti (stran 23).
            \end{proof}


            \begin{zgled}
                Naj bo $G: \R\rightarrow\R$ narascajoca in zvezno odvedljiva. Potem za realne $a \leq b$ velja $(G'\cdot\mathcal{L})((a, b]) = \int_{(a, b]}G'd\mathcal{L} = \int_a^b G'(x)dx = G(b) - G(a)$. Po definiciji je torej $G'\cdot\mathcal{L} = dG.$ in $\int gdG = \int gd(G'\cdot\mathcal{L}) = \int gG'd\mathcal{L}$ za $\forall g \in \mathcal{B}_\R\backslash\mathcal{B}_{[-\infty, \infty]}$ za katere je $\int gG'd\mathcal{L}$ dobro definiran.
            \end{zgled}

            \begin{trditev}
                Naj bo $(X, \A, \mu)$ prostor z mero in $\{f, g\} \subset \A\backslash\B_{[-\infty, \infty]}.$
                \begin{enumerate}
                    \item Naj bo $\int_{\{f > g\}}f^+d\mu \vee \int_{\{f > g\}}g^-d\mu < \infty$ in $\int_{\{f > g\}}fd\mu \leq \int_{\{f > g\}}gd\mu < \infty$. Potem je $f \leq g$ s.p.-$\mu$.
                    \item Naj bo $\mu$-koncna, $\int fd\mu$ in $\int gd\mu$  dobro definirana in naj bo $\int_{A}fd\mu \leq \int_{A}gd\mu \ \forall A \in \A$. Potem je $f \leq g$ s.p.-$\mu$.
                \end{enumerate}
            \end{trditev}

            \begin{proof}
                Brez dokaza.
            \end{proof}

            \begin{posledica}
                Naj bo $(X, \A, \mu)$ prostor z mero in $\{f, g\} \subset \A\backslash\B_{[-\infty, \infty]}.$ Denimo, da je $(\star)$ $\int fd\mu = \int gd\mu$ za $\forall A \in \A$, pri cemer sta $\int$ dobro definirana. Ce je 
                \begin{itemize}
                    \item $f$ ali $g$ $\mu$-integrabilna (in torej obe $\mu$-integrabilni) ali 
                    \item $\mu$ je $\sigma$-koncna,
                \end{itemize}
                je $f = g$ s.p.-$\mu$. \\
                Nadalje v primeru $\{f, g\} \subset \Le^1(\mu)$ je namesto $(\star)$ dovolj zahtevati $\int_Afd\mu = \int_Agd\mu \ \forall A \in \Pi\cup\{X\}$. za nek $\pi$-sistem $\Pi$, ki generira $\A$. 
            \end{posledica}

            \begin{proof}
                
            \end{proof}

            \begin{izrek}(Radon-Nikodyn)
                Naj bo $(\Omega, \F)$ merljiv prostor in $\mu$ ter $\nu$ $\sigma$-koncni meri na njem in $\mu << \nu$. Potem obstaja $f \in \F\backslash\mathcal{B}_{[-\infty, \infty]} \ni \mu = f\cdot\nu$. Nadalje je $f>0$ s.p-$\mu$. Taka funkcija je dolocena s.p.-$\nu$ natancno. 
            \end{izrek}

            \begin{proof}
                Dokaz enolicnosti:\\
                Ce imamo $\{f_1, f_2\} \subset \F\backslash\mathcal{B}_{[0, \infty)}, f_1\cdot\nu = \mu = f_2\cdot\nu$, potem iz prejsnje posledice sledi $f_1 = f_2$ s.p-$\nu$ \\
                Dokaz $f>0$ s.p-$\mu$: \\
                $\mu(\{f = 0\}) = (f\cdot\nu)(\{f = 0\}) = \int_{\{f = 0\}}fd\nu = \int\underbrace{\mathds{1}_{\{f = 0\}}}_{=0}d\nu = 0$ Torej je $f>0$ s.p-$\mu$. 
            \end{proof}

            \begin{definicija}
                Vsaki funkciji $f$ iz izreka pravimo Radon-Nikodynov odvod $\mu$ glede na $\nu$ in jo oznacimo z $\frac{d\mu}{d\nu}$.
            \end{definicija}

            \begin{posledica}
                Naj bodo $\mu, \nu, \lambda$ $\sigma$-koncne mere na $\sigma$-algebri $\F$ in $\mu << \nu << \lambda$. Potem je $\frac{d\mu}{d\lambda} = \frac{d\mu}{d\nu}\frac{d\nu}{d\lambda}$ s.p-$\lambda$. Posebej: $\mu \sim \nu \Rightarrow \frac{d\mu}{d\nu}\frac{d\nu}{d\mu} = 1$ s.p-$\mu$ in s.p-$\nu$.
            \end{posledica}

            \begin{proof}
                $$
                \left(\frac{d\mu}{d\nu}\frac{d\nu}{d\lambda}\right)\cdot\lambda = 
                \frac{d\mu}{d\nu}\cdot\left(\underbrace{\frac{d\nu}{d\lambda}\cdot\lambda}_{=\nu}\right) =
                \frac{d\mu}{d\nu}\cdot\nu = \mu
                $$
                Torej je $\frac{d\mu}{d\nu}\frac{d\nu}{d\lambda} = \frac{d\mu}{d\lambda}$ s.p-$\lambda$. Ce je $\mu \sim \nu$ lahko v enakosti vzamemo $\lambda = \mu$, potem je $\frac{d\mu}{d\nu}\frac{d\nu}{d\mu} = \frac{d\mu}{d\mu} = 1$ s.p-$\mu$ in s.p-$\nu$.
            \end{proof}

        \subsubsection{L-prostori in integralske neenakosti}
            \begin{definicija}
                Naj bo $(\Omega, \F, \mu)$ prostor z mero in $p\in[1, \infty).$ Za $f \in \F\backslash\mathcal{B}_{[-\infty, \infty]}$ definiramo
                $$
                    ||f||_{p_\mu} := \left(\int |f|^pd\mu\right)^{\frac{1}{p}}
                $$
                in vpeljemo $ \mathcal{L}^p(\mu) := \{f \in \F\backslash\mathcal{B}_{\R} \mid ||f||_{p_\mu} < \infty\}.$
                Dodatno definiramo za $f \in \F\backslash\mathcal{B}_{[-\infty, \infty]}$
                $$
                    ||f||_{\infty_\mu} := \inf\{M \in [0, \infty] \mid |f| < M \ s.p.-\mu\}
                $$
                In postavimo $\mathcal{L}^\infty(\mu) := \{f \in \F\backslash\mathcal{B}_{\R} \mid ||f||_{\infty_\mu} < \infty\}.$ \\
                Za $p\in[1, \infty]$ in za zaporedje $(f_n)_{n\in\N_0}$ v $\mathcal{L}^p(\mu)$ pravimo, da 
                $$
                f_n \xrightarrow{n \rightarrow \infty}{} f_0 \ \text{v} \ \Le^p(\mu) \iff ||f_n - f_0||_{p_\mu} \xrightarrow{n \rightarrow \infty}{} 0.
                $$
                V tem primeru pravimo, da je $f_0$ limita $(f_n)_{n\in\N}$ v $\mathcal{L}^p(\mu)$ in pisemo $f_0 = \lim_{n\rightarrow\infty}f_n$ v $\mathcal{L}^p(\mu)$.
                

            \end{definicija}

\end{document}