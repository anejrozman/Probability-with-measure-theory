\documentclass[a4paper,12pt]{article}
\usepackage[margin=3cm]{geometry}
\usepackage[slovene]{babel}
\usepackage[utf8]{inputenc}
\usepackage[T1]{fontenc}
\usepackage{lmodern}
\usepackage{url}
\usepackage{graphicx}
\usepackage{amsmath}
\usepackage{amssymb}
\usepackage{amsfonts}
\usepackage{hyperref}
\usepackage{amsthm}
\usepackage{pgfpages}
\usepackage{colortbl}
\usepackage{tikz}
\usepackage{array}
\usepackage{amsmath,amsthm, amsfonts,amssymb}
\usepackage{mathtools}
\usepackage{tikz}
\usepackage{dsfont}


\newtheorem{definicija}{Definicija}
\newtheorem{trditev}{Trditev}
\newtheorem{izrek}{Izrek}
\newtheorem{posledica}{Posledica}
\newtheorem{lema}{Lema}
\newtheorem{zgled}{Zgled}
\newtheorem{opomba}{Opomba}

\newcommand{\R}{\mathbb{R}}
\newcommand{\N}{\mathbb{N}}
\newcommand{\E}{\mathbb{E}}
\newcommand{\F}{\mathcal{F}}
\newcommand{\A}{\mathcal{A}}
\newcommand{\sh}{š}
\newcommand{\ch}{č}
\newcommand{\zh}{ž}
\newcommand{\SH}{Š}
\newcommand{\CH}{Č}
\newcommand{\ZH}{Ž}




\begin{document}

\begin{titlepage}
    UNIVERZA V LJUBLJANI
  
    FAKULTETA ZA MATEMATIKO IN FIZIKO
  
    \vspace{0.5cm}
    Finančna matematika - 1. stopnja
  
    \begin{center}
        \vspace{7cm}
            Anej Rozman
  
        \vspace{0.4cm}
        \textbf{\Large{Verjetnost z mero}}
        \vspace{0.3cm}
  
        Zapiski po predavanjih doc.\@ dr.\@ Matije Vidmarja v študijskem letu 2023/2024.
    \end{center}
    \vfill
        Ljubljana, 2023     
    \thispagestyle{empty}
\end{titlepage}

\newpage
  
\tableofcontents
   
\newpage
    
\section{Mera}
    \subsection{Merljivost in mere}
        \subsubsection{Merjlive mnozice}
            \begin{definicija}
                Naj bo $\mathcal{A} \subset 2^\Omega$, torej $\mathcal{A} \in 2^{2^\Omega}$ Pravimo, da je $\mathcal{A}$ zaprta za:
                \begin{enumerate}
                    \item $c^\Omega$ (zaprta za komplemente v $\Omega$) $\iff \forall A \in \mathcal{A} \Rightarrow \Omega \backslash A \in \mathcal{A} $
                    \item $\cap $ (zaprta za preseke) $\iff A \cap A' \in \mathcal{A}$ kadarkoli $\{A, A'\} \subset \mathcal{A}$
                    \item $\cup$ (zaprta za unije) $iff A \cup A' \in \mathcal{A}$ kadarkoli $\{A, A'\} \subset \mathcal{A}$
                    \item $\backslash$ (zaprta za razlike) $\iff A \backslash A' \in \mathcal{A}$ kadarkoli $\{A, A'\} \subset \mathcal{A}$
                    \item $\sigma\cap$ (zaprta za stevne preseke) $\iff \forall (A_n)_{n \in \N} \in \mathcal{A}$ kadarkoli je $\left(A_n \right)_{n \in \N}$ zaporedje v $\mathcal{A}$
                    \item $\sigma\cup$ (zaprta za stevne unije) $\iff \forall (A_n)_{n \in \N} \in \mathcal{A} $ kadarkoli je $\left(A_n \right)_{n \in \N}$ zaporedje v $\mathcal{A}$
                
                \end{enumerate} 
            \end{definicija}
            
            \begin{definicija}($\sigma$-algebra, pod-$\sigma$-algebra in algebra)
                \begin{enumerate}
                    \item$\mathcal{A}$ je $\sigma$-algebra na $\Omega \iff (\Omega, A)$ je merljiv prostor $\iff \emptyset \in \mathcal{A}$ in $\mathcal{A}$ je zaprta za $c^\Omega$ in $\sigma\cup$. 
                    \item Ce je $\mathcal{A} \sigma$-algebra na $\Omega$ potem: $A$ je $\mathcal{A}$-merljiva $\iff A \in \mathcal{A}; \mathcal{B}$ je pod-$\sigma$-algebra $\mathcal{A} \iff \mathcal{B} \subset \mathcal{A}$ in $\mathcal{B}$ je $\sigma$-algebra na $\Omega$.
                    \item $\mathcal{A}$ je algebra na $\Omega \iff \emptyset \in \mathcal{A}$ in $\mathcal{A}$ je zaprta za $c^\Omega$ in $\cup$.
                \end{enumerate}
            \end{definicija}


        \begin{zgled}
            $2^\Omega$ je $\sigma$-algebra na $\Omega$ in $\{\emptyset, \Omega\}$ je $\sigma$-algebra na $\Omega$. Klicemo ju diskretna in trivialna $\sigma$-algebra.
        \end{zgled}

        \begin{zgled}
        Naj bo $A \subset \Omega$ Potem je $ \sigma_\Omega A:=\{\emptyset, A, A^C, \Omega\}$ $\sigma$-algebra na $\Omega$.
        \end{zgled}

        \begin{zgled}
        $\sigma_\Omega^{ccc} := \{ A \in 2^\Omega: A \ \text{je stevna ali} \ \Omega \backslash A \ \text{je stevna}\}$ je $\sigma$-algebra na $\Omega$.
        To je oznaka za stevno kostevno $\sigma$-algebro na $\Omega$. Seveda je $\sigma_\Omega^{ccc}$ = $2^\Omega$ razen ce $\Omega$ ni stevna.
        \end{zgled}

        \begin{zgled}
        Naj bo $\mathcal{P}$ particija $\Omega$ (torej $\mathcal{P} \subset 2^\Omega$ in $\mathcal{P}$ je druzina paroma disjunktnih mnozic, ki pokrije $\Omega$).
        Potem je $\sigma \mathcal{P} := \{\cup R \mid R \subset P \ in \ (R \ ali \ P\backslash R \ je \ stevna)\}$ je sigma algebra na $\Omega$.
        \end{zgled}

        \begin{trditev}
            Naj bo $\A \subset 2^\Omega$ zaprta za $c^\Omega$ in naj bo $\emptyset \in \A$. Potem je $\A$ $\sigma$-algebra na $\Omega$ ce in samo ce je $\A$ zaprta za $\sigma\cup$, in v tem primeru je $\A$ zaprta za $\cap$ in $\cup$ in $\backslash$.
        \end{trditev}
        \begin{proof}
            Sledi iz de Morganovih zakonov:
            $$
            \cap_{n \in \N} A_n = \Omega \backslash \left(\cup_{n \in \N}(\Omega\backslash A_n) \right)
            $$
            $$
            \cup_{n \in \N} A_n = \Omega \backslash \left(\cap_{n \in \N}(\Omega\backslash A_n) \right)
            $$
            Zaprtost $\sigma$-algebre $A$ na $\Omega$ za:
            \begin{enumerate}
                \item $\cap$: $A \cap B = A \cap B \cap \Omega \cap \Omega \cdots$
                \item $\cup$: $A \cup B = A \cup B \cup \emptyset \cup \emptyset \cdots$
                \item $\backslash$: $A \backslash B = A  \cap \left( \Omega \backslash B \right) \in A$
            \end{enumerate}

        \end{proof}

        \subsubsection{Mere}
            \begin{definicija}(Mera)
                Naj bo $\left( \Omega, \F \right)$ merljiv prostor in $\mu: \F \rightarrow [0, \infty] $. \\
                $\mu$ je mera na $\left( \Omega, \F \right)$ natanko tedaj ko: 
                \begin{enumerate}
                    \item $\mu(\emptyset) = 0$
                    \item $\mu$ je stevno aditivna: $\forall (A_n)_{n \in \N} \subset \F$ paroma disjunktnih mnozic je $\mu(\cup_{n \in \N} A_n) = \sum_{n \in \N} \mu(A_n)$
                \end{enumerate}
            \end{definicija}

            Za mero $\mu$ na $\left( \Omega, F \right)$ recemo da je: 
            \begin{enumerate}
                \item koncna $\iff \mu(\Omega) < \infty$
                \item verjetnostna mera $\iff \mu(\Omega) = 1$
                \item $\sigma$-koncna $\iff \exists (A_n)_{n \in \N} \subset \F: \Omega = \cup_{m \in \N} A_m \ \text{in} \ \mu(A_n) < \infty \ \forall n \in \N$
            \end{enumerate}

            \begin{definicija}(Merljiv prostor)
                $\left( \Omega, F, \mu \right)$ je prostor z mero $\iff \mu$ je mera na $(\Omega, F)$
            \end{definicija}

            \begin{definicija}
                Ce je $\left( \Omega, F, \mu \right)$ merljiv prostor. Potem za $A \in \F$ recemo:
                \begin{enumerate}
                    \item $A$ je $\mu$-zanemarljiva $\iff \mu(A) = 0$
                    \item $A$ je $\mu$-trivialna $\iff \mu(A) = 0$ ali $\mu(\Omega\backslash A) = 0$
                \end{enumerate}
            \end{definicija}

            Ce imamo neko lastnost $P(\omega)$ v $\omega \in A$, potem: \\
            \begin{enumerate}
                \item $P(\omega)$ drzi $\mu$-skoraj povsod v $\omega \in A \iff A_{\neg P} := \{\omega \in A: \neg P(\omega)\}$ je $\mu$-zanemarljiva.       
                \item $P(\omega)$ drzi $\mu$-skoraj gotovo v $\omega \in A \iff A_{\neg P} := \{\omega \in A: \neg P(\omega)\}$ je $\mu$-zanemarljiva in $\mu$ je verjetnost.
                \item $P$ drzi $\mu$-s.p. na $A \iff P(\omega)$ drzi $\mu$-s.p. v $\omega \in A$. 
                \item $P$ drzi $\mu$-s.g. na $A \iff P(\omega)$ drzi $\mu$-s.g. v $\omega \in A$. 
            \end{enumerate}
            \begin{zgled}
                Nicelna mera na $\F$ torej preslikava $\mu(A) \rightarrow 0 \ \forall A \in \F$ je vedno mera na katerikoli $\sigma$-algebri.
            \end{zgled}      
            
            \begin{zgled}
                Ce definiramo $c_\Omega:2^\Omega \rightarrow [0, \infty]$ kot $c_\Omega(A) := |A|$ ce je $A$ koncna podmnozica $\Omega$ in $c_\Omega(A) := \infty$ ca je neskoncna podmnozica $\Omega$, je $c_\Omega$ tako imenovana stevna mera na $\Omega$. Ko je $\Omega$ koncna in neprazna, potem je $\frac{c_\Omega}{|\Omega|}$ verjentostna mera na $\Omega$.  
            \end{zgled}

            \begin{zgled}
                Ce definiramo $\delta_x:2^\Omega \rightarrow [0, \infty]$ za fiksen $x \in \Omega$, tako da za $A \in 2^\Omega, \delta_x(A):=0$ ce $x \notin A$ in $\delta_x(A):=1$ ce $x \in A$, potem je $\delta_x$ tako imenovana Diracova mera za $x$. Katerakoli podmnozica $\Omega \backslash \{x\}$ je $\delta_x$-zanemarljiva.                
            \end{zgled}
            
            \begin{trditev}(Lastnosti mere)
                Naj bo $\mu$ mera na merljivem prostoru $\left( \Omega, \F \right)$. Potem velja naslednje:
                \begin{enumerate}
                    \item $\mu$ je aditivna: $\mu(A \cup B) = \mu(A) + \mu(B)$ ce velja $A \cap B = \emptyset$ in $\{A, B\} \in \F$.
                    \item $\mu$ je monotona: $A \subset B$ in $ \{A, B\} \in \F$ $ \Rightarrow \mu(A) \leq \mu(B)$.
                    \item $\mu$ je zvezna od spodaj: $\mu(\cup_{n \in \N}A_n) = \uparrow\text{-}\lim_{n \rightarrow \infty}\mu(A_n)$ kadarkoli je $(A_n)_{n \in \N}$ narascajoce zaporedje v $\F$.
                    \item $\mu$ je stevno subaditivna: $\mu(\cup_{n \in \N}A_n) \leq \sum_{n \in \N}\mu(A_n)$ kadarkoli je $(A_n)_{n \in \N}$ zaporedje v $\F$.
                    \item Predpostavimo da je $\mu$ koncna. $\mu(\Omega\backslash A) = \mu(\Omega) - \mu(A)$ za vse $A \in \F$. Se vec, $\mu$ je zvezna od zgoraj: $\mu(\cap_{n \in \N}A_n) = \downarrow\text{-}\lim_{n \rightarrow \infty}\mu(A_n)$ kadarkoli je $(A_n)_{n \in \N}$ padajoce zaporedje v $\F$.
                    \item Za $A \in \F$ je $\F|_A:=\{B \cap A: B \in \F\}$; potem je $\mu_A:= \mu_{\F|_A}$ mera na $\F|_A$. Imenuje se restrikcija/skrcitev mere $\mu$ na $A$.  
                \end{enumerate}             
            \end{trditev}

            \begin{proof}
                \begin{enumerate}
                    \item $\left(A, B, \emptyset, \emptyset, \cdots \right)$ je zaporedje medseboj disjunktnih mnozic v $\F$.
                    \item $B = A \cup (B \backslash A)$ in uporabimo koncno aditivnost (1.).
                    \item $\left(A_1, A_2\backslash A_1, A_3\backslash A_2, \cdots \right)$ je zaporedje paroma disjunktnih mnozic v $\F$ z unijo $\cup_{n \in \N}A_n$. Uporabimo stevno aditivnost in za tem se koncno aditivnost.
                    \item $\left(A_1, A_2\backslash A_1, A_3\backslash (A_1 \cup A_2), \cdots \right)$ je zaporedje paroma disjunktnih mnozic v $\F$ z unijo $\cup_{n \in \N}A_n$. Uporabimo stevno aditivnost in za tem se monotonost (2.).
                    \item Prvi del sledi iz koncne aditivnosti (1.). Za mnozici vzamemo $A \in \Omega$ in  $B = \Omega \backslash A$. Dobimo $\mu(\Omega) = \mu(A) + \mu(\Omega \backslash A)$. Drugi del sledi iz zveznosti od spodaj (3.) tako da jo uporabimo na $\left(\Omega\backslash A_n \right)_{n \in \N}$.
                    \item Preveriti moramo, da je $\F|_A$ $\sigma$-algebra na $A$. Kasneje bomo videli da je $\F|_A = 2^A \cap \F$ in bo dokaz sledil iz tega.
                \end{enumerate}
            \end{proof}

            \begin{zgled}(Borel-Cantellijeva lema)
                Naj bo $\left( \Omega, \F, \mu \right)$ merljiv prostor in $(A_n)_{n \in \N}$ zaporedje v $\F$, da velja $\sum_{n \in \N}\mu(A_n) < \infty$. Potem je $\mu(\limsup_{n \rightarrow \infty}A_n) = 0$.
            \end{zgled}
            \begin{zgled}
                Ce je $P$ vejretnost na $\left( \Omega, \F \right)$, potem je $P^{-1}(\{0, 1\})$ pod-$\sigma$-algebra na $\F$. Tako imenovana $P$-trivialna $\sigma$-algebra.
            \end{zgled}

        \subsubsection{Merjlive preslikave in generirane $\sigma$-algebre}
            
            \begin{definicija}(Generirana $\sigma$-algebra)
                Naj bo $\A \subset 2^\Omega$; potem $$\sigma_\Omega(\A):=\cap\{\F \in 2^{2^\Omega}: \F \ \text{je} \ \sigma\text{-algebra na} \ \Omega \ \text{in} \ \A \subset \F \}$$ imenujemo $\sigma$-algebra generirana na $\Omega$ z $\A$.
            \end{definicija}

            \begin{opomba}
                $2^\Omega$ je gotovo $\sigma$-algebra na $\Omega$, ki vsebuje $\A$, torej je $\sigma_\Omega(\A)$ neprazna.
            \end{opomba}

            Za dve $\sigma$-algebri $\mathcal{B}_1$ in $\mathcal{B}_2$ na $\Omega$ je mnozica $\mathcal{B}_1 \vee \mathcal{B}_2 := \sigma_\Omega\left(\mathcal{B}_1 \cup \mathcal{B}_2\right)$ zdruzitev $\mathcal{B}_1$ in $\mathcal{B}_2$. Bolj splosno
            za druzino $\left(\mathcal{B}_\lambda\right)_{\lambda \in \Lambda} \sigma$-algeber na $\Omega$ pravimo $\vee_{\lambda \in \Lambda}\mathcal{B}_\lambda := \sigma_\Omega\left(\cup_{\lambda \in \Lambda}\mathcal{B}_\lambda\right)$ druzina druzin.

            \begin{opomba}
                Razlog zakaj so generirane $\sigma$-algebre pomembne v teoriji mere je, ker le redko lahko eksplicitno
                podamo vse elemente $\sigma$-algebre, ki bi jo zeleli, ampak pogosto lahko eksplicitno podamo njene generatorje.  
            \end{opomba}

            \begin{definicija}(Zacetna in koncna struktura)
                Naj bo $f:\Omega \rightarrow \Omega'$. Za podano $\sigma$-algebro $\F'$ na $\Omega'$  deifniramo $$\sigma^{\F'}(f):= f^{-1}(\F'):=\{f^{-1}(A'):A' \in \F'\},$$
                zacetno strukturo za $f$ glede na $\F'$. (oziroma $\sigma$-algebra generirana z $f$ glede na $\F'$).

                Za podano $\sigma$-algebro $\F$ na $\Omega$ definiramo $$\sigma^{\Omega'}_{\F}(f):=\{A \in 2^{\Omega'}: f^{-1}(A') \in \F\},$$
                koncno strukturo za $f$ na $\Omega'$ glede na $\F$.
            \end{definicija}

            \begin{definicija}
                Za dano $\sigma$-algebro $\F'$ na $\Omega'$ in $\sigma$-algebro $\F$ na $\Omega$ pravimo da je $f$ $\F/\F'$-merljiva preslikava $\iff f^{-1}(A')\in \F$ za vse $A' \in \F'$.
            \end{definicija}

            \begin{opomba}
                \begin{enumerate}
                    \item V notacijah $\sigma_\Omega(\A)$ in $\sigma^{\F'}(f)$ spuscamo $\Omega$ in $\F'$ kadar je to jasno iz konteksta. Torej pisemo preprosto $\sigma(\A)$ in $\sigma(f)$.
                    \item V primeru ko je zaloga vrednosti $f$ stevna in nimamo podane ne $\F'$ ali $\Omega'$ za $\Omega'$ vazamemo $Z_f$ in za $\F'$ vazamemo $2^{\Omega'}$.
                    \item Izmed objektov, ki smo ju uvedli v definiciji 7 je zacetna struktura veliko bolj pomembna
                \end{enumerate}
            \end{opomba}

            \begin{definicija}
                Za dano $\sigma$-algebro $\F$ na $\Omega$ in $\F'$ na $\Omega'$ definiramo $$\F/\F':= \{g \in \Omega'^\Omega: g \ \text{je} \ \F/\F'\text{-merljiva}\}$$.
            \end{definicija}

            \begin{zgled}
                Konstantna funkcija je vedno merljiva, ne glede na $\sigma$-algebro. Za poljubno $\sigma$-algebro $\F$ na $\Omega$ je $id_\Omega \in \F/\F'$.
            \end{zgled}

            \begin{definicija}(Indikator)
                Za $A \in \Omega$ definiramo $\mathds{1}_{A_\Omega}:\Omega\rightarrow\{0, 1\}$:
                $$\mathds{1}_{A_\Omega}(x) := 
                    \begin{cases}
                        1, & \ x \in A \\
                        0, & \text{sicer}
                    \end{cases}$$
                Funkciji pravimo indikator mnozice $A$ z underlying prostorom $\Omega$. Pisali bomo $\mathds{1}_A$ in predvidevali da se $\Omega$ da razbrati iz konteksta.
            \end{definicija}

            \begin{zgled}
                Naj bo $A \subset \Omega$. Potem $\sigma^{2^{\{0, 1\}}}(\mathds{1}_A) = \sigma_\Omega A$. Ce je nadaljno $\F$ $\sigma$-algebra an $\Omega$ potem je  $\mathds{1}_A \in \F/2^{\{0, 1\}} \iff A \in \F$ 
            \end{zgled}

            \begin{trditev}
                Naj bodo $\F, \mathcal{G}, \mathcal{H}$ $\sigma$-algebre (vsaka na svoji mnozici). Naj bo $f \in \F/\mathcal{G}$ in $g \in \mathcal{G}/\mathcal{H}$. Potem je $g \circ f \in \F/\mathcal{H}.$ Z besedami to pomeni: kompozitumi merljivih preslikav so merljive preslikave.
            \end{trditev}

            \begin{proof}
                $(g \circ f)^{-1}(H) = f^{-1}(g^{-1}(H))$ za $H$ $\in \mathcal{H}$
            \end{proof}

            \begin{trditev} (Lastnosti preslikav)
                Naj bo $f: \Omega \rightarrow \Omega'$.
                \begin{enumerate}
                    \item Naj bo $\F'$ $\sigma$-algebra na $\Omega'$. $\sigma^{\F'}(f)$ je $\sigma$-algebra na $\Omega$; je najmanjsa $\sigma$-algebra $\mathcal{G}$ na $\Omega$ da velja $f \in \mathcal{G}/\F'$.
                    \item Naj bo $\F$ $\sigma$-algebra na $\Omega$. $\sigma_{\F}^{\Omega'}(f)$ je $\sigma$-algebra na $\Omega'$; je najvecja $\sigma$-algebra $\mathcal{G}'$ na $\Omega'$ da velja $f \in \mathcal{G}'/\F'$.
                    \item Naj bo $\F'$ $\sigma$-algebra na $\Omega'$ in $\F$ $\sigma$-algebra na $\Omega$, potem $f \in \F/\F' \iff \sigma^{\F'}(f) \subset \F \iff \sigma_{\F}^{\Omega'}(f) \supset \F'.$
                    \item Naj bo $\A' \subset 2^{\Omega'}$. $\sigma_{\Omega'}(\A')$ je najmanjsa $\sigma$-algebra na $\Omega'$, ki ima $\A'$ za svojo podmnozico. Naj bo $\F$ $\sigma$-algebra na $\Omega$, potem je $f \in \F/\sigma_{\Omega'}(\A') \iff (f^{-1}(A') \in \F$ za vse $A' \in \A'$). Natancno zapisemo $\sigma^{\sigma_{\Omega'}(\A')}(f) = \sigma_\Omega(\{f^{-1}(A'):A' \in \A'\})$
                \end{enumerate}
            \end{trditev}

            \begin{proof}
                \begin{enumerate}
                    \item nothing
                \end{enumerate}
            \end{proof}

            \begin{opomba}
                Tocka 4 nam pove da je dovolj da dokazemo  merljivo lastnost na mnozici generatorjev. Se en nacin zapisa $f \in \F/\sigma_{\Omega'}(\A') \iff (f^{-1}(A') \in \F$ za vse $A' \in \A'$). Natancno zapisemo $\sigma^{\sigma_{\Omega'}(\A')}(f) = \sigma_\Omega(\{f^{-1}(A'):A' \in \A'\})$ je $f^{-1}(\sigma_{\Omega'}(\A')) = \sigma_\Omega(f^{-1}(\A'))$, kar bomo interpretirali lpt operacija dobivanja praslik in generiranih $\sigma$-algeber komutirati.
            \end{opomba}

            \begin{definicija}
                Pisemo $\A|_A := \{A'\cap A : A'\in \A\}$ za sled $\A$ na $A$.
            \end{definicija}
\end{document}